\documentclass[a4paper,punct=custom/kaiming,fontset=none]{ctexart}

\newcommand*{\titleContent}{双重否定的演绎}

\title{\bfseries\titleContent}

\usepackage{mathtools,amssymb,amsthm}
\usepackage{lualatex-math}

\setmainfont{Times New Roman}[BoldFont=Heiti SC,Ligatures=Rare]
\setCJKmainfont{Songti SC}[BoldFont=Heiti SC,ItalicFont=Kaiti SC]
\usepackage[
  math-style=TeX,
  warnings-off={mathtools-colon,mathtools-overbracket}
]{unicode-math}
\setmathfont{Asana Math}
% \setmathfont{STIX Two Math}[range="002C]

\newcommand*{\enumparen}[1]{\textnormal{(}\makebox[0.5em][c]{#1}\textnormal{)}}

% https://www.logicmatters.net/resources/nd3.sty
% https://www.logicmatters.net/resources/pdfs/nd-manual2a.pdf
\usepackage{nd3}
\renewcommand*{\ndstretch}{.25}

\usepackage{xcolor}
\usepackage[
  pdfusetitle,
  hyperfootnotes=false,
  colorlinks=true,
  urlcolor={[rgb]{0,.2,.6}},
  linkcolor={[rgb]{0,.2,.6}}
]{hyperref}

\makeatletter
\newcommand*{\theH@NDlines}{\@NDident.\number\value{@NDlines}}
\newtheoremstyle{remark}{3pt}{3pt}{}{}{\itshape}{\ignorespaces}{\thm@headsep}{}
\renewenvironment{proof}[1][\proofname]{\par
  \pushQED{\qed}%
  \normalfont \topsep6\p@\@plus6\p@\relax
  \trivlist
  % \item[]\ignorespaces
  \item[\hskip\labelsep
    \itshape
    #1%
    % \@addpunct{\hskip6.5pt}%
    ]\ignorespaces
}{%
  \popQED\endtrivlist\@endpefalse
}
\def\@maketitle{%
  \newpage
  \null
  \vskip 2em%
  \begin{center}%
    \let \footnote \thanks
    {\LARGE \@title \par}%
  \end{center}%
  \par
  \vskip 1.5em}
\makeatother

\theoremstyle{remark}
\newtheorem*{remark}{评注}

\AtBeginDocument{%
  % \renewcommand{\perp}{\mathrel{\bot}}
  \let\oldref\ref
  \let\proves\vdash
  \let\leq\leqslant
  \let\le\leq
  \let\geq\geqslant
  \let\ge\geq}

\begin{document}
\maketitle

\noindent \(\neg\neg\phi \proves \phi\)
\begin{proof}
  \leavevmode
  \begin{ND}[][dn]
    \ndl{}{\(\neg\neg\phi\)}{假设}\label{1}
    \ndl{}{\(\neg\phi \to \neg\phi\)}{示例2.4.8}\label{2}
    \ndl{}{\((\neg\phi \to \neg\phi) \to (\neg\phi \to \neg\neg\phi) \to \phi\)}{\(\rm P_3\)}\label{3}
    \ndl{}{\(\neg\neg\phi \to \neg\phi \to \neg\neg\phi\)}{\(\rm P_1\)}\label{4}
    \ndl{}{\(\neg\phi \to \neg\neg\phi\)}{\ref{1},\ref{4},MP}\label{5}
    \ndl{}{\((\neg\phi \to \neg\neg\phi) \to \phi\)}{\ref{2},\ref{3},MP}\label{6}
    \ndl{}{\(\phi\)}{\ref{5},\ref{6},MP}
  \end{ND}
\end{proof}
\begin{remark}
  上述的第\ndref{dn}{4}、\ndref{dn}{5}步可以合并成一步,原因列就写“\enumparen{\ndref{dn}{1}},示例2.4.18”。
\end{remark}

\end{document}

% Local Variables:
% TeX-engine: luatex
% End:
