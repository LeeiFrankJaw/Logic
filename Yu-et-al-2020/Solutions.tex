\newcommand*{\titleContent}{《数理逻辑》习题}
\newcommand*{\authorContent}{赵磊}
\newif\ifbsixpaper
% \bsixpapertrue
\ifbsixpaper
\documentclass[punct=custom/kaiming,fontset=none]{ctexart}
\usepackage[b6paper,landscape,hmargin=.4in,vmargin=.3in]{geometry}
\title{\Large\bfseries\titleContent}
\else
\documentclass[a4paper,punct=custom/kaiming,fontset=none]{ctexart}
\usepackage[hmargin=1in,vmargin=1in]{geometry}
\title{\bfseries\titleContent}
\fi

\author{\authorContent}

\usepackage{mathtools,amssymb,amsthm}
\newcommand\SetSymbol[1][]{%
  \nonscript\:#1\vert
  \allowbreak
  \nonscript\:
  \mathopen{}}
% \DeclarePairedDelimiterX{\paren}[1]{\lparen}{\rparen}{%
%   \renewcommand{\given}{\SetSymbol[\delimsize]}#1}
\DeclarePairedDelimiterX{\brce}[1]{\lbrace}{\rbrace}{%
  \newcommand*{\given}{\SetSymbol[\delimsize]}#1}
\DeclarePairedDelimiterXPP{\Set}[1]{\mathop{}}{\lbrace}{\rbrace}{}{%
  \newcommand{\given}{\SetSymbol[\delimsize]}#1}
\usepackage{lualatex-math}

\newif\ifadobe
% \adobetrue

\ifadobe
\setmainfont{Latin Modern Roman}[BoldFont=Adobe Heiti Std]
\setCJKmainfont{Adobe Song Std}[
  BoldFont=Adobe Heiti Std,
  ItalicFont=Adobe Kaiti Std
]
\else
\setmainfont{Times New Roman}[BoldFont=Heiti SC,Ligatures=Rare]
\setCJKmainfont{Songti SC}[BoldFont=Heiti SC,ItalicFont=Kaiti SC]
\setCJKmainfont{FZShuSong-Z01}[CharRange={"00B7}]
\usepackage[
  math-style=TeX,
  warnings-off={mathtools-colon,mathtools-overbracket}
]{unicode-math}
\setmathfont{Asana Math}
% \setmathfont{STIX Two Math}[range="002C]
\fi

\newcommand*{\enumparen}[1]{\textnormal{(}\makebox[0.6em][c]{#1}\textnormal{)}}

\usepackage{enumitem}
\setlist[enumerate,1]{label=\enumparen{\arabic*}}
\setlist[enumerate,2]{label=\enumparen{\alph*}}

\usepackage{float}

% https://www.logicmatters.net/resources/nd3.sty
% https://www.logicmatters.net/resources/pdfs/nd-manual2a.pdf
\usepackage{nd3}
\renewcommand*{\ndstretch}{.25}
\renewcommand*{\ndrulestretch}{1.1}

\usepackage{xcolor}
\usepackage{graphicx}
\usepackage[
  pdftitle={\titleContent},
  pdfauthor={\authorContent},
  hyperfootnotes=false,
  colorlinks=true,
  urlcolor={.},
  linkcolor={.}
]{hyperref}

\newcommand*{\vdashv}{\mathrel{\ooalign{$\vdash$\cr\hskip1pt\reflectbox{$\vdash$}}}}

\makeatletter
\def\hhline{%
  \noalign{\ifnum0=`}\fi\hrule \@height 2\arrayrulewidth \futurelet
   \reserved@a\@xhline}
\newcommand*{\theH@NDlines}{\@NDident.\number\value{@NDlines}}
\renewenvironment{proof}[1][\proofname]{\par
  \pushQED{\qed}%
  \normalfont \topsep6\p@\@plus6\p@\relax
  \trivlist
  % \item[]\ignorespaces
  \item[\hskip\labelsep
    \bfseries
    #1%
    % \@addpunct{:}%
    ]\ignorespaces
}{%
  \popQED\endtrivlist\@endpefalse
}
\newtheoremstyle{plain}{3pt}{3pt}{}{}{\bfseries}{}{\thm@headsep}{%
  \thmname{#1}\thmnumber{\@ifnotempty{#1}{ }\@upn{#2}}%
  \thmnote{ {\the\thm@notefont(#3)}}%
  % \@ifnotempty{#3}{\thm@headsep=.75ex}
}
\newtheoremstyle{remark}{3pt}{3pt}{}{}{\itshape}{}{\thm@headsep}{}
\makeatother

\newtheorem*{lemma*}{引理}
\theoremstyle{remark}
\newtheorem*{remark}{评注}

\setlength{\jot}{0pt}

\AtBeginDocument{%
  \setlength{\abovedisplayskip}{.8ex}
  \setlength{\belowdisplayskip}{.8ex}
  \let\asserts⊦
  \let\proves\vdash
  \let\leq\leqslant
  \let\le\leq
  \let\geq\geqslant
  \let\ge\geq
}

\begin{document}
\maketitle

此文档为《数理逻辑》(余俊伟 等,2020)一书的非官方习题解答。

\begin{description}
\item[习题 2.2.6] 前三项都是公式,剩余六项都不是公式。
\item[习题 2.2.7]
  \begin{enumerate}
  \item[]
  \item 补全括号:
    \begin{enumerate}
    \item \(((r \vee (\neg p)) \vee q)\);
    \item \(((r \to (p \to p)) \leftrightarrow ((\neg(\neg p)) \vee q))\)。
    \end{enumerate}
  \item 简化公式:
    \begin{enumerate}
    \item 题有问题,多一个左括号,或者少一个右括号。假设是后者,就有\(p \wedge q \to r\);
    \item \(p \to q\);
    \item \((r \to p \to q) \to s\)。
    \end{enumerate}
  \end{enumerate}
\item[问题 2.3.8] \(16\)和\(2^n\)。
\item[习题 2.3.12] 因为前面讨论的原子命题个数是有穷的(finite),而且现在讨论的情况是有可数无穷多个(countably infinite)原子命题。
\item[习题 2.3.21]
  \begin{enumerate}
  \item[]
  \item 由下表可知,它是重言式。
    \begin{table}[H]
      \centering
      \begin{tabular}[t]{ccccc}
        \hhline
        \(p\) & \(q\) & \(p \to q\) & \(p \land (p \to q)\) & \(p \land (p \to q) \to q\) \\
        \hline
        1 & 1 & 1 & 1 & 1 \\
        1 & 0 & 0 & 0 & 1 \\
        0 & 1 & 1 & 0 & 1 \\
        0 & 0 & 1 & 0 & 1 \\
        \hline
      \end{tabular}
    \end{table}
  \item 由下表可知,它是矛盾式。
    \begin{table}[H]
      \centering
      \begin{tabular}[t]{cccccc}
        \hhline
        \(p\) & \(q\) & \(p \to q\) & \((p \to q) \to p\) & \(\neg p\) & \(((p \to q) \to p) \land \neg p\) \\
        \hline
        1 & 1 & 1 & 1 & 0 & 0 \\
        1 & 0 & 0 & 1 & 0 & 0 \\
        0 & 1 & 1 & 0 & 1 & 0 \\
        0 & 0 & 1 & 0 & 1 & 0 \\
        \hline
      \end{tabular}
    \end{table}
  \end{enumerate}
\item[习题 2.3.22] \(v(\neg p \land (q \to r)) = 0\)。
\item[习题 2.3.23] 对于任意的赋值\(v\),要么\(v(p) = 1\),要么\(v(p) = 0\)。在前者的情况下,有\(v(\neg p) = 0\),从而\(v(p \land \neg p) = 0\)。同理,在后者的情况下,也有\(v(p \land \neg p) = 0\)。所以它是矛盾式。\qed
\item[习题 2.3.24]
  \begin{enumerate}
  \item[]
  \item 对于任意的赋值\(v\),要么\(v(\phi) = 1\),要么\(v(\phi) = 0\)。若是前者,则\(v(\psi \to \phi) = 1\),自然有\(v(\phi \to \psi \to \phi) = 1\)。同理,若是后者,则直接有\(v(\phi \to \psi \to \phi) = 1\)。因此,它是重言式。\qed
  \item 对于任意的赋值\(v\),要么\(v(\phi \to \theta) = 1\),要么\(v(\phi \to \theta) = 0\)。若是前者,则有\(v((\phi \to \psi) \to (\phi \to \theta)) = 1\),从而\(v((\phi \to \psi \to \theta) \to (\phi \to \psi) \to (\phi \to \theta)) = 1\)。若是后者,则有\(v(\phi) = 1\)且\(v(\theta) = 0\)。此时,若\(v(\psi) = 1\),则有\(v(\phi \to \psi \to \theta) = 0\),从而\(v((\phi \to \psi \to \theta) \to (\phi \to \psi) \to (\phi \to \theta)) = 1\);若\(v(\psi) = 0\),则有\(v(\phi \to \psi) = 0\),从而有\(v((\phi \to \psi) \to (\phi \to \theta)) = 1\)和\(v((\phi \to \psi \to \theta) \to (\phi \to \psi) \to (\phi \to \theta)) = 1\)。因此,它是重言式。\qed
  \item 对于任意的赋值\(v\),要么\(v(\phi) = 1\),要么\(v(\phi) = 0\)。若是前者,则自然有\(v((\neg\phi \to \neg\psi) \to \phi) = 1\),从而\(v((\neg\phi \to \psi) \to (\neg\phi \to \neg\psi) \to \phi) = 1\)。若是后者,则\(v(\neg\phi) = 1\)。此时,若\(v(\psi) = 1\),则\(v(\neg\psi) = 0\),从而有\(v(\neg\phi \to \neg\psi) = 0\)、\(v((\neg\phi \to \neg\psi) \to \phi) = 1\)和\(v((\neg\phi \to \psi) \to (\neg\phi \to \neg\psi) \to \phi) = 1\);若\(v(\psi) = 0\),则\(v(\neg\phi \to \psi) = 0\),从而\(v((\neg\phi \to \psi) \to (\neg\phi \to \neg\psi) \to \phi) = 1\)。因此,它是重言式。\qed
  \end{enumerate}
\item[习题 2.3.25] 设\(v\)为任意满足\(v(\Set{p \to q, \neg p \to \neg q}) = 1\)的赋值函数。因为\(v(p \to q) = 1\),所以\(v(p) = 0\)或者\(v(q) = 1\)。若\(v(p) = 0\),则\(v(\neg p) = 1\)。又因为\(v(\neg p \to \neg q) = 1\),所以有\(v(\neg q) = 1\),从而有\(v(q) = 0\)和\(v(p \leftrightarrow q) = 1\)。若\(v(q) = 1\),则\(v(\neg q) = 0\)。又因为\(v(\neg p \to \neg q) = 1\),所以有\(v(\neg p) = 0\),从而有\(v(p) = 1\)和\(v(p \leftrightarrow q) = 1\)。由\(v\)的任意性可得\(\Set{p \to q, \neg p \to \neg q} \models p \leftrightarrow q\)。\qed
\item[习题 2.4.7]
  \begin{enumerate}
  \item[]
  \item 根据定义2.4.4中的\enumparen{1}即可得证。\qed
  \item 因为\(\phi_1, \dotsc, \phi_n\)是一个证明,所以它满足定义2.4.4。那么,对于任意的\(1 \le j \le k \le n\),也都有\(\phi_j\)满足定义2.4.4。因此,\(\phi_1, \dotsc, \phi_k\)也是一个证明。\qed
  \end{enumerate}
\item[问题 2.4.13] 证明可以看成以\(\emptyset\)为假设的一个演绎。
\item[定理 2.4.21] 下面,我们先证明一些引定理和引演绎。我们把它们统称为引理,可以看到这些引理的使用能让后续证明的结构清晰明了。
  \begin{lemma*}
    \label{ded:comm}
    \(\phi \to \psi \to \theta \proves \psi \to \phi \to \theta\)\hfill(交换演绎)
    \begin{proof}
      \leavevmode
      \begin{ND}
        \ndl{}{\(\phi \to \psi \to \theta\)}{假设}\label{1}
        \ndl{}{\((\phi \to \psi \to \theta) \to (\phi \to \psi) \to (\phi \to \theta)\)}{\(\rm P_2\)}\label{2}
        \ndl{}{\(\psi \to (\phi \to \psi)\)}{\(\rm P_1\)}\label{3}
        \ndl{}{\((\phi \to \psi) \to (\phi \to \theta)\)}{\ref{1},\ref{2},MP}\label{4}
        \ndl{}{\(\psi \to \phi \to \theta\)}{\ref{3},\ref{4},示例2.4.19}
      \end{ND}
    \end{proof}
  \end{lemma*}
  \begin{lemma*}
    \label{ded:condMp}
    \(\Set{\phi \to \psi, \phi \to \psi \to \theta} \proves \phi \to \theta\)\hfill(条件MP)
    \begin{proof}
      \leavevmode
      \begin{ND}[][][][\rwidth{lem:dnElim}]
        \ndl{}{\(\phi \to \psi\)}{假设}\label{1}
        \ndl{}{\(\phi \to \psi \to \theta\)}{假设}\label{2}
        \ndl{}{\((\phi \to \psi \to \theta) \to (\phi \to \psi) \to (\phi \to \theta)\)}{\(\rm P_2\)}\label{3}
        \ndl{}{\((\phi \to \psi) \to (\phi \to \theta)\)}{\ref{2},\ref{3},MP}\label{4}
        \ndl{}{\(\phi \to \theta\)}{\ref{1},\ref{4},MP}
      \end{ND}
    \end{proof}
  \end{lemma*}
  \begin{lemma*}
    \label{lem:mp}
    \(\proves \phi \to (\phi \to \psi) \to \psi\)\hfill(分离引理)
    \begin{proof}
      \leavevmode
      \begin{ND}
        \ndl{}{\((\phi \to \psi) \to (\phi \to \psi)\)}{示例2.4.8}\label{1}
        \ndl{}{\(((\phi \to \psi) \to (\phi \to \psi)) \to ((\phi \to \psi) \to \phi) \to ((\phi \to \psi) \to \psi)\)}{\(\rm P_2\)}\label{2}
        \ndl{}{\(\phi \to ((\phi \to \psi) \to \phi)\)}{\(\rm P_1\)}\label{3}
        \ndl{}{\(((\phi \to \psi) \to \phi) \to ((\phi \to \psi) \to \psi)\)}{\ref{1},\ref{2},MP}\label{4}
        \ndl{}{\(\phi \to (\phi \to \psi) \to \psi\)}{\ref{3},\ref{4},示例2.4.19}
      \end{ND}
    \end{proof}
  \end{lemma*}
  \begin{lemma*}
    \label{lem:condRepDist}
    \(\proves (\psi \to \theta) \to (\phi \to \psi) \to (\phi \to \theta)\)\hfill(双件引入引理)
    \begin{proof}
      \leavevmode
      \begin{ND}[][][][\rwidth{lem:dnElim}]
        \ndl{}{\((\psi \to \theta) \to \phi \to (\psi \to \theta)\)}{\(\rm P_1\)}\label{1}
        \ndl{}{\((\phi \to \psi \to \theta) \to (\phi \to \psi) \to (\phi \to \theta)\)}{\(\rm P_2\)}\label{2}
        \ndl{}{\((\psi \to \theta) \to (\phi \to \psi) \to (\phi \to \theta)\)}{\ref{1},\ref{2},MP}
      \end{ND}
    \end{proof}
  \end{lemma*}
  \begin{lemma*}
    \label{ded:condRepDist}
    \(\psi \to \theta \proves (\phi \to \psi) \to (\phi \to \theta)\)\hfill(双件引入)
    \begin{proof}
      \leavevmode
      \begin{ND}[][][][\rwidth{lem:dnElim}]
        \ndl{}{\(\psi \to \theta\)}{假设}\label{1}
        \ndl{}{\((\psi \to \theta) \to (\phi \to \psi) \to (\phi \to \theta)\)}{\hyperref[lem:condRepDist]{双件引入引理}}\label{2}
        \ndl{}{\((\phi \to \psi) \to (\phi \to \theta)\)}{\ref{1},\ref{2},MP}
      \end{ND}
    \end{proof}
  \end{lemma*}
  \begin{lemma*}
    \label{lem:trans}
    \(\proves (\phi \to \psi) \to (\psi \to \theta) \to (\phi \to \theta)\)\hfill(传递引理)
    \begin{proof}
      \leavevmode
      \begin{ND}[][][][\rwidth{lem:dnElim}]
        \ndl{}{\((\psi \to \theta) \to (\phi \to \psi) \to (\phi \to \theta)\)}{\hyperref[lem:condRepDist]{双件引入引理}}\label{1}
        \ndl{}{\((\phi \to \psi) \to (\psi \to \theta) \to (\phi \to \theta)\)}{\ref{1},\hyperref[ded:comm]{交换演绎}}
      \end{ND}
    \end{proof}
  \end{lemma*}
  \begin{lemma*}
    \label{lem:dnElim}
    \(\proves \neg\neg\phi \to \phi\)\hfill(双重否定消去引理)
    \begin{proof}
      \leavevmode
      \begin{ND}[][lem:dnElim]
        \ndl{}{\(\neg\phi \to \neg\phi\)}{示例2.4.8}\label{1}
        \ndl{}{\((\neg\phi \to \neg\phi) \to (\neg\phi \to \neg\neg\phi) \to \phi\)}{\(\rm P_3\)}\label{2}
        \ndl{}{\(\neg\neg\phi \to (\neg\phi \to \neg\neg\phi)\)}{\(\rm P_1\)}\label{3}
        \ndl{}{\((\neg\phi \to \neg\neg\phi) \to \phi\)}{\ref{1},\ref{2},MP}\label{4}
        \ndl{}{\(\neg\neg\phi \to \phi\)}{\ref{3},\ref{4},示例2.4.19}
      \end{ND}
    \end{proof}
  \end{lemma*}
  \begin{lemma*}
    \label{lem:dnIntro}
    \(\proves \phi \to \neg\neg\phi\)\hfill(双重否定引入引理)
    \begin{proof}
      \leavevmode
      \begin{ND}
        \ndl{}{\(\phi \to (\neg\neg\neg\phi \to \phi)\)}{\ref{1},示例2.4.18}\label{1}
        \ndl{}{\((\neg\neg\neg\phi \to \phi)
          \to (\neg\neg\neg\phi \to \neg\phi)
          \to \neg\neg\phi\)}{\(\rm P_3\)}\label{2}
        \ndl{}{\(\phi \to (\neg\neg\neg\phi \to \neg\phi) \to \neg\neg\phi\)}{\ref{1},\ref{2},示例2.4.19}\label{4}
        \ndl{}{\(\neg\neg\neg\phi \to \neg\phi\)}{\hyperref[lem:dnElim]{双重否定消去引理}}\label{5}
        \ndl{}{\((\neg\neg\neg\phi \to \neg\phi) \to \phi \to \neg\neg\phi\)}{\ref{4},\hyperref[ded:comm]{交换演绎}}\label{6}
        \ndl{}{\(\phi \to \neg\neg\phi\)}{\ref{5},\ref{6},MP}
      \end{ND}
    \end{proof}
  \end{lemma*}
  \begin{lemma*}
    \label{lem:dnIntroAnte}
    \(\proves (\phi \to \psi) \to (\neg\neg\phi \to \psi)\)\hfill(前件双重否定引理)
    \begin{proof}
      \leavevmode
      \begin{ND}[][][][\rwidth{lem:dnElim}]
        \ndl{}{\(\neg\neg\phi \to \phi\)}{\hyperref[lem:dnElim]{双重否定消去引理}}\label{1}
        \ndl{}{\((\neg\neg\phi \to \phi) \to (\phi \to \psi) \to (\neg\neg\phi \to \psi)\)}{\hyperref[lem:trans]{传递引理}}\label{2}
        \ndl{}{\((\phi \to \psi) \to (\neg\neg\phi \to \psi)\)}{\ref{1},\ref{2},MP}
      \end{ND}
    \end{proof}
  \end{lemma*}
  % \begin{lemma*}
  %   \label{ded:dnIntroAnte}
  %   \(\phi \to \psi \proves \neg\neg\phi \to \psi\)\hfill(前件双重否定)
  %   \begin{proof}
  %     \leavevmode
  %     \begin{ND}
  %       \ndl{}{\(\phi \to \psi\)}{假设}\label{1}
  %       \ndl{}{\((\phi \to \psi) \to (\neg\neg\phi \to \psi)\)}{\hyperref[lem:dnIntroAnte]{前件双重否定引理}}\label{2}
  %       \ndl{}{\(\neg\neg\phi \to \psi\)}{\ref{1},\ref{2},MP}
  %     \end{ND}
  %   \end{proof}
  % \end{lemma*}
  \begin{lemma*}
    \label{lem:p3var}
    \(\proves (\phi \to \psi) \to (\phi \to \neg\psi) \to \neg\phi\)\hfill(\(\rm P_3\)变体)
    \begin{proof}
      \leavevmode
      \begin{ND}
        \ndl{}{\((\phi \to \neg\psi) \to (\neg\neg\phi \to \neg\psi)\)}{\hyperref[lem:dnIntroAnte]{前件双重否定引理}}\label{1}
        \ndl{}{\((\neg\neg\phi \to \neg\psi) \to (\neg\neg\phi \to \psi) \to \neg\phi\)}{\(\rm P_3\),\hyperref[ded:comm]{交换演绎}}\label{2}
        \ndl{}{\((\phi \to \neg\psi) \to (\neg\neg\phi \to \psi) \to \neg\phi\)}{\ref{1},\ref{2},示例2.4.19}\label{3}
        \ndl{}{\((\phi \to \psi) \to (\neg\neg\phi \to \psi)\)}{\hyperref[lem:dnIntroAnte]{前件双重否定引理}}\label{4}
        \ndl{}{\((\neg\neg\phi \to \psi) \to (\phi \to \neg\psi) \to \neg\phi\)}{\ref{3},\hyperref[ded:comm]{交换演绎}}\label{5}
        \ndl{}{\((\phi \to \psi) \to (\phi \to \neg\psi) \to \neg\phi\)}{\ref{4},\ref{5},示例2.4.19}
      \end{ND}
    \end{proof}
  \end{lemma*}
  \begin{lemma*}
    \label{lem:contrapose1}
    \(\proves (\neg\phi \to \neg\psi) \to (\psi \to \phi)\)\hfill(假言易位引理1)
    \begin{proof}
      \leavevmode
      \begin{ND}
        \ndl{}{\(\psi \to (\neg\phi \to \psi)\)}{\(\rm P_1\)}\label{1}
        \ndl{}{\((\neg\phi \to \psi) \to (\neg\phi \to \neg\psi) \to \phi\)}{\(\rm P_3\)}\label{2}
        \ndl{}{\(\psi \to (\neg\phi \to \neg\psi) \to \phi\)}{\ref{1},\ref{2},示例2.4.19}\label{3}
        \ndl{}{\((\neg\phi \to \neg\psi) \to (\psi \to \phi)\)}{\ref{3},\hyperref[ded:comm]{交换演绎}}
      \end{ND}
    \end{proof}
  \end{lemma*}
  \begin{lemma*}
    \label{lem:contrapose2}
    \(\proves (\neg\phi \to \psi) \to (\neg\psi \to \phi)\)\hfill(假言易位引理2)
    \begin{proof}
      \leavevmode
      \begin{ND}
        \ndl{}{\(\neg\psi \to (\neg\phi \to \neg\psi)\)}{\(\rm P_1\)}\label{1}
        \ndl{}{\((\neg\phi \to \neg\psi) \to (\neg\phi \to \psi) \to \phi\)}{\(\rm P_3\),\hyperref[ded:comm]{交换演绎}}\label{2}
        \ndl{}{\(\neg\psi \to (\neg\phi \to \psi) \to \phi\)}{\ref{1},\ref{2},示例2.4.19}\label{3}
        \ndl{}{\((\neg\phi \to \psi) \to (\neg\psi \to \phi)\)}{\ref{3},\hyperref[ded:comm]{交换演绎}}
      \end{ND}
    \end{proof}
  \end{lemma*}
  \begin{lemma*}
    \label{lem:contrapose3}
    \(\proves (\phi \to \neg\psi) \to (\psi \to \neg\phi)\)\hfill(假言易位引理3)
    \begin{proof}
      \leavevmode
      \begin{ND}
        \ndl{}{\((\phi \to \neg\psi) \to (\neg\neg\phi \to \neg\psi)\)}{\hyperref[lem:dnIntroAnte]{前件双重否定引理}}\label{1}
        \ndl{}{\((\neg\neg\phi \to \neg\psi) \to (\psi \to \neg\phi)\)}{\hyperref[lem:contrapose1]{假言易位引理1}}\label{2}
        \ndl{}{\((\phi \to \neg\psi) \to (\psi \to \neg\phi)\)}{\ref{1},\ref{2},示例2.4.19}
      \end{ND}
    \end{proof}
  \end{lemma*}
  \begin{lemma*}
    \label{lem:exFalso}
    \(\proves \phi \to \neg\phi \to \psi\)\hfill(爆炸原理)
    \begin{proof}
      \leavevmode
      \begin{ND}
        \ndl{}{\(\phi \to (\neg\psi \to \phi)\)}{\(\rm P_1\)}\label{1}
        \ndl{}{\((\neg\psi \to \phi) \to (\neg\phi \to \psi)\)}{\hyperref[lem:contrapose2]{假言易位引理2}}\label{2}
        \ndl{}{\(\phi \to \neg\phi \to \psi\)}{\ref{1},\ref{2},示例2.4.19}
      \end{ND}
      % 或者
      % \begin{ND}
      %   \ndl{}{\(\phi \to (\neg\psi \to \phi)\)}{\(\rm P_1\)}\label{1}
      %   \ndl{}{\((\neg\psi \to \phi) \to (\neg\psi \to \neg\phi) \to \psi\)}{\(\rm P_3\)}\label{2}
      %   \ndl{}{\(\phi \to (\neg\psi \to \neg\phi) \to \psi\)}{\ref{1},\ref{2},示例2.4.19}\label{3}
      %   \ndl{}{\(\neg\phi \to (\neg\psi \to \neg\phi)\)}{\(\rm P_1\)}\label{4}
      %   \ndl{}{\((\neg\psi \to \neg\phi) \to \phi \to \psi\)}{\ref{3},\hyperref[ded:comm]{交换演绎}}\label{5}
      %   \ndl{}{\(\neg\phi \to \phi \to \psi\)}{\ref{4},\ref{5},示例2.4.19}\label{6}
      %   \ndl{}{\(\phi \to \neg\phi \to \psi\)}{\ref{6},\hyperref[ded:comm]{交换演绎}}
      % \end{ND}
    \end{proof}
  \end{lemma*}
  \begin{lemma*}
    \label{lem:conjIntro}
    \(\proves \phi \to \psi \to \phi \land \psi\)\hfill(合取引入引理)
    \begin{proof}
      \leavevmode
      \begin{ND}
        \ndl{}{\(\psi \to ((\phi \to \neg\psi) \to \psi)\)}{\(\rm P_1\)}\label{1}
        \ndl{}{\(((\phi \to \neg\psi) \to \psi) \to ((\phi \to \neg\psi) \to \neg\psi) \to \neg(\phi \to \neg\psi)\)}{\hyperref[lem:p3var]{\(\rm P_3\)变体}}\label{2}
        \ndl{}{\(\psi \to ((\phi \to \neg\psi) \to \neg\psi) \to \neg(\phi \to \neg\psi)\)}{\ref{1},\ref{2},示例2.4.19}\label{3}
        \ndl{}{\(\phi \to ((\phi \to \neg\psi) \to \neg\psi)\)}{\hyperref[lem:mp]{分离引理}}\label{4}
        \ndl{}{\(((\phi \to \neg\psi) \to \neg\psi) \to \psi \to \neg(\phi \to \neg\psi)\)}{\ref{3},\hyperref[ded:comm]{交换演绎}}\label{5}
        \ndl{}{\(\phi \to \psi \to \neg(\phi \to \neg\psi)\)}{\ref{4},\ref{5},示例2.4.19}\label{6}
        \ndl{}{\(\phi \to \psi \to \phi \land \psi\)}{\ref{6},定义2.2.4}
      \end{ND}
    \end{proof}
  \end{lemma*}
  \begin{lemma*}
    \label{lem:conjElimLeft}
    \(\proves \phi \land \psi \to \psi\)\hfill(合取左消去引理)
    \begin{proof}
      \leavevmode
      \begin{ND}
        \ndl{}{\(\neg\psi \to (\phi \to \neg\psi)\)}{\(\rm P_1\)}\label{1}
        \ndl{}{\((\neg\psi \to (\phi \to \neg\psi))
          \to (\neg\psi \to \neg(\phi \to \neg\psi))
          \to \psi\)}{\(\rm P_3\)}\label{2}
        \ndl{}{\(\neg(\phi \to \neg\psi) \to (\neg\psi \to \neg(\phi \to \neg\psi))\)}{\(\rm P_1\)}\label{3}
        \ndl{}{\((\neg\psi \to \neg(\phi \to \neg\psi)) \to \psi\)}{\ref{1},\ref{2},MP}\label{4}
        \ndl{}{\(\neg(\phi \to \neg\psi) \to \psi\)}{\ref{3},\ref{4},示例2.4.19}\label{5}
        \ndl{}{\(\phi \land \psi \to \psi\)}{\ref{5},定义2.2.4}
      \end{ND}
    \end{proof}
  \end{lemma*}
  \begin{lemma*}
    \label{lem:conjElimRight}
    \(\proves \phi \land \psi \to \phi\)\hfill(合取右消去引理)
    \begin{proof}
      \leavevmode
      \begin{ND}
        \ndl{}{\(\neg\phi \to (\phi \to \neg\psi)\)}{\hyperref[lem:exFalso]{爆炸原理},\hyperref[ded:comm]{交换演绎}}\label{1}
        \ndl{}{\((\neg\phi \to (\phi \to \neg\psi))
          \to (\neg\phi \to \neg(\phi \to \neg\psi))
          \to \phi\)}{\(\rm P_3\)}\label{2}
        \ndl{}{\(\neg(\phi \to \neg\psi) \to (\neg\phi \to \neg(\phi \to \neg\psi))\)}{\(\rm P_1\)}\label{3}
        \ndl{}{\((\neg\phi \to \neg(\phi \to \neg\psi)) \to \phi\)}{\ref{1},\ref{2},MP}\label{4}
        \ndl{}{\(\neg(\phi \to \neg\psi) \to \phi\)}{\ref{3},\ref{4},示例2.4.19}\label{5}
        \ndl{}{\(\phi \land \psi \to \phi\)}{\ref{5},定义2.2.4}
      \end{ND}
    \end{proof}
  \end{lemma*}
  \begin{lemma*}
    \label{lem:negTrans}
    \(\proves \neg(\phi \to \psi) \to (\theta \to \psi) \to \neg(\phi \to \theta)\)\hfill(否定传递引理)
    \begin{proof}
      \leavevmode
      \begin{ND}[][][][\rwidth{lem:dnElim}]
        \ndl{}{\((\theta \to \psi) \to (\phi \to \theta) \to (\phi \to \psi)\)}{\hyperref[lem:condRepDist]{双件引入引理}}\label{1}
        \ndl{}{\(
          \begin{aligned}[t]
            &((\phi \to \theta) \to (\phi \to \psi))
              \to ((\phi \to \theta) \to \neg(\phi \to \psi)) \\
            &\quad\to \neg(\phi \to \theta)
          \end{aligned}
          \)}{\hyperref[lem:p3var]{\(\rm P_3\)变体}}\label{2}
        \ndl{}{\((\theta \to \psi)
          \to ((\phi \to \theta) \to \neg(\phi \to \psi))
          \to \neg(\phi \to \theta)\)}{\ref{1},\ref{2},示例2.4.19}\label{3}
        \ndl{}{\(\neg(\phi \to \psi) \to ((\phi \to \theta) \to \neg(\phi \to \psi))\)}{\(\rm P_1\)}\label{4}
        \ndl{}{\(((\phi \to \theta) \to \neg(\phi \to \psi))
          \to (\theta \to \psi)
          \to \neg(\phi \to \theta)\)}{\ref{3},\hyperref[ded:comm]{交换演绎}}\label{5}
        \ndl{}{\(\neg(\phi \to \psi) \to (\theta \to \psi) \to \neg(\phi \to \theta)\)}{\ref{4},\ref{5},示例2.4.19}
      \end{ND}
    \end{proof}
  \end{lemma*}
  \begin{lemma*}
    \label{lem:dm1}
    \(\proves \neg(\phi \lor \psi) \to (\neg\phi \land \neg\psi)\)\hfill(德·摩根引理1)
    \begin{proof}
      \leavevmode
      \begin{ND}[][lem:dm1][][\rwidth{lem:dnElim}]
        \ndl{}{\(\neg(\neg\phi \to \psi)
          \to (\neg\neg\psi \to \psi)
          \to \neg(\neg\phi \to \neg\neg\psi)\)}{\hyperref[lem:negTrans]{否定传递引理}}\label{1}
        \ndl{}{\(\neg\neg\psi \to \psi\)}{\hyperref[lem:dnElim]{双重否定消去引理}}\label{2}
        \ndl{}{\((\neg\neg\psi \to \psi)
          \to \neg(\neg\phi \to \psi)
          \to \neg(\neg\phi \to \neg\neg\psi)\)}{\ref{1},\hyperref[ded:comm]{交换演绎}}\label{3}
        \ndl{}{\(\neg(\neg\phi \to \psi)
          \to \neg(\neg\phi \to \neg\neg\psi)\)}{\ref{2},\ref{3},MP}\label{4}
        \ndl{}{\(\neg(\phi \lor \psi) \to (\neg\phi \land \neg\psi)\)}{\ref{4},定义2.2.4}
      \end{ND}
    \end{proof}
  \end{lemma*}
  \begin{lemma*}
    \label{lem:dm2}
    \(\proves (\neg\phi \land \neg\psi) \to \neg(\phi \lor \psi)\)\hfill(德·摩根引理2)
    \begin{proof}
      实际上,只需将前一个证明中的第\ndref{lem:dm1}{2}步改成\(\psi \to \neg\neg\psi\),其他步骤再做相应的修改即可,如下。
      \begin{ND}[][][][\rwidth{lem:dnElim}]
        \ndl{}{\(\neg(\neg\phi \to \neg\neg\psi)
          \to (\psi \to \neg\neg\psi)
          \to \neg(\neg\phi \to \psi)\)}{\hyperref[lem:negTrans]{否定传递引理}}\label{1}
        \ndl{}{\(\psi \to \neg\neg\psi\)}{\hyperref[lem:dnIntro]{双重否定引入引理}}\label{2}
        \ndl{}{\((\psi \to \neg\neg\psi)
          \to \neg(\neg\phi \to \neg\neg\psi)
          \to \neg(\neg\phi \to \psi)\)}{\ref{1},\hyperref[ded:comm]{交换演绎}}\label{3}
        \ndl{}{\(\neg(\neg\phi \to \neg\neg\psi)
          \to \neg(\neg\phi \to \psi)\)}{\ref{2},\ref{3},MP}\label{4}
        \ndl{}{\((\neg\phi \land \neg\psi) \to \neg(\phi \lor \psi)\)}{\ref{4},定义2.2.4}
      \end{ND}
    \end{proof}
  \end{lemma*}
  \begin{enumerate}
  \item \(\phi \proves \neg\neg\phi\)\hfill(\hypertarget{ded:dnIntro}{双重否定引入})
    \begin{proof}
      \leavevmode
      \begin{ND}[][][][\rwidth{ded:conjAssoc2}]
        \ndl{}{\(\phi\)}{假设}\label{1}
        \ndl{}{\(\phi \to \neg\neg\phi\)}{\hyperref[lem:dnIntro]{双重否定引入引理}}\label{2}
        \ndl{}{\(\neg\neg\phi\)}{\ref{1},\ref{2},MP}
      \end{ND}
      % \begin{ND}[][][][\rwidth{ded:conjAssoc2}]
      %   \ndl{}{\(\phi\)}{假设}\label{1}
      %   \ndl{}{\(\neg\neg\neg\phi \to \phi\)}{\ref{1},示例2.4.18}\label{2}
      %   \ndl{}{\((\neg\neg\neg\phi \to \phi)
      %     \to (\neg\neg\neg\phi \to \neg\phi)
      %     \to \neg\neg\phi\)}{\(\rm P_3\)}\label{3}
      %   \ndl{}{\(\neg\neg\neg\phi \to \neg\phi\)}{\hyperref[lem:dnElim]{双重否定消去引理}}\label{4}
      %   \ndl{}{\((\neg\neg\neg\phi \to \neg\phi)
      %     \to \neg\neg\phi\)}{\ref{2},\ref{3},MP}\label{5}
      %   \ndl{}{\(\neg\neg\phi\)}{\ref{4},\ref{5},MP}
      % \end{ND}
    \end{proof}
    \(\neg\neg\phi \proves \phi\)\hfill(\hypertarget{ded:dnElim}{双重否定消去})
    \begin{proof}
      \leavevmode
      \begin{ND}[][][][\rwidth{ded:conjAssoc2}]
        \ndl{}{\(\neg\neg\phi\)}{假设}\label{1}
        \ndl{}{\(\neg\neg\phi \to \phi\)}{\hyperref[lem:dnElim]{双重否定消去引理}}\label{2}
        \ndl{}{\(\phi\)}{\ref{1},\ref{2},MP}
      \end{ND}
    \end{proof}
  \item 蕴含引入和蕴含消去分别见示例2.4.18和2.4.17。
  \item \(\Set{\phi, \psi} \proves \phi \land \psi\)\hfill(\hypertarget{ded:conjIntro}{合取引入})
    \begin{proof}
      \leavevmode
      \begin{ND}[][][][\rwidth{ded:conjAssoc2}]
        \ndl{}{\(\phi\)}{假设}\label{1}
        \ndl{}{\(\psi\)}{假设}\label{2}
        \ndl{}{\(\phi \to \psi \to \phi \land \psi\)}{\hyperref[lem:conjIntro]{合取引入引理}}\label{3}
        \ndl{}{\(\psi \to \phi \land \psi\)}{\ref{1},\ref{3},MP}\label{4}
        \ndl{}{\(\phi \land \psi\)}{\ref{2},\ref{4},MP}
      \end{ND}
      % 或者
      % \begin{ND}[][][][\rwidth{ded:conjAssoc2}]
      %   \ndl{}{\(\phi\)}{假设}\label{1}
      %   \ndl{}{\(\psi\)}{假设}\label{2}
      %   \ndl{}{\((\phi \to \neg\psi) \to \psi\)}{\ref{2},示例2.4.18}\label{3}
      %   \ndl{}{\(((\phi \to \neg\psi) \to \psi)
      %     \to ((\phi \to \neg\psi) \to \neg\psi)
      %     \to \neg(\phi \to \neg\psi)\)}{\hyperref[lem:p3var]{\(\rm P_3\)变体}}\label{4}
      %   \ndl{}{\(\phi \to (\phi \to \neg\psi) \to \neg\psi\)}{\hyperref[lem:mp]{分离引理}}\label{5}
      %   \ndl{}{\((\phi \to \neg\psi) \to \neg\psi\)}{\ref{1},\ref{5},MP}\label{6}
      %   \ndl{}{\(((\phi \to \neg\psi) \to \neg\psi)
      %     \to \neg(\phi \to \neg\psi)\)}{\ref{3},\ref{4},MP}\label{7}
      %   \ndl{}{\(\neg(\phi \to \neg\psi)\)}{\ref{6},\ref{7},MP}\label{8}
      %   \ndl{}{\(\phi \land \psi\)}{\ref{8},定义2.2.4}
      % \end{ND}
      或者
      \begin{ND}[][][][\rwidth{ded:conjAssoc2}]
        \ndl{}{\(\phi\)}{假设}\label{1}
        \ndl{}{\(\psi\)}{假设}\label{2}
        \ndl{}{\(\neg\neg(\phi \to \neg\psi) \to \psi\)}{\ref{2},示例2.4.18}\label{3}
        \ndl{}{\(
          \begin{aligned}[t]
            &(\neg\neg(\phi \to \neg\psi) \to \psi)
              \to (\neg\neg(\phi \to \neg\psi) \to \neg\psi) \\
            &\quad\to \neg(\phi \to \neg\psi)
          \end{aligned}
          \)}{\(\rm P_3\)}\label{4}
        \ndl{}{\(\neg\neg(\phi \to \neg\psi) \to (\phi \to \neg\psi)\)}{%
          \hyperref[lem:dnElim]{双重否定消去引理}}\label{5}
        \ndl{}{\(
          \begin{aligned}[t]
            &(\neg\neg(\phi \to \neg\psi) \to (\phi \to \neg\psi))
              \to (\neg\neg(\phi \to \neg\psi) \to \phi) \\
            &\quad\to (\neg\neg(\phi \to \neg\psi) \to \neg\psi)
          \end{aligned}
          \)}{\(\rm P_2\)}\label{6}
        \ndl{}{\(\neg\neg(\phi \to \neg\psi) \to \phi\)}{%
          \ref{1},示例2.4.18}\label{7}
        \ndl{}{\((\neg\neg(\phi \to \neg\psi) \to \phi)
          \to (\neg\neg(\phi \to \neg\psi) \to \neg\psi)\)}{%
          \ref{5},\ref{6},MP}\label{8}
        \ndl{}{\(\neg\neg(\phi \to \neg\psi) \to \neg\psi\)}{%
          \ref{7},\ref{8},MP}\label{9}
        \ndl{}{\((\neg\neg(\phi \to \neg\psi) \to \neg\psi)
          \to \neg(\phi \to \neg\psi)\)}{\ref{3},\ref{4},MP}\label{10}
        \ndl{}{\(\neg(\phi \to \neg\psi)\)}{\ref{9},\ref{10},MP}\label{11}
        \ndl{}{\(\phi \land \psi\)}{\ref{11},定义2.2.4}
      \end{ND}
    \end{proof}
    \(\phi \land \psi \proves \phi\)\hfill(\hypertarget{ded:conjElim}{合取消去})
    \begin{proof}
      \leavevmode
      \begin{ND}[][][][\rwidth{ded:conjAssoc2}]
        \ndl{}{\(\phi \land \psi\)}{假设}\label{1}
        \ndl{}{\(\phi \land \psi \to \phi\)}{\hyperref[lem:conjElimRight]{合取右消去引理}}\label{2}
        \ndl{}{\(\phi\)}{\ref{1},\ref{2},MP}
      \end{ND}
      % 或者
      % \begin{ND}[][ded:conjElim][][\rwidth{ded:conjAssoc2}]
      %   \ndl{}{\(\phi \land \psi\)}{假设}\label{1}
      %   \ndl{}{\(\neg(\phi \to \neg\psi)\)}{\ref{1},定义2.2.4}\label{2}
      %   \ndl{}{\(\neg\phi \to (\phi \to \neg\psi)\)}{\hyperref[lem:exFalso]{爆炸原理},\hyperref[ded:comm]{交换演绎}}\label{3}
      %   \ndl{}{\((\neg\phi \to (\phi \to \neg\psi))
      %     \to (\neg\phi \to \neg(\phi \to \neg\psi))
      %     \to \phi\)}{\(\rm P_3\)}\label{4}
      %   \ndl{}{\(\neg\phi \to \neg(\phi \to \neg\psi)\)}{\ref{2},示例2.4.18}\label{5}
      %   \ndl{}{\((\neg\phi \to \neg(\phi \to \neg\psi)) \to \phi\)}{\ref{3},\ref{4},MP}\label{6}
      %   \ndl{}{\(\phi\)}{\ref{5},\ref{6},MP}
      % \end{ND}
    \end{proof}
  \item \(\phi \proves \phi \lor \psi\)\hfill(析取引入)
    \begin{proof}
      \leavevmode
      \begin{ND}[][][][\rwidth{ded:conjAssoc2}]
        \ndl{}{\(\phi\)}{假设}\label{1}
        \ndl{}{\(\phi \to \neg\phi \to \psi\)}{\hyperref[lem:exFalso]{爆炸原理}}\label{2}
        \ndl{}{\(\neg\phi \to \psi\)}{\ref{1},\ref{2},MP}\label{3}
        \ndl{}{\(\phi \lor \psi\)}{\ref{3},定义2.2.4}
      \end{ND}
    \end{proof}
    \pagebreak[2]
    \(\Set{\phi \lor \psi, \phi \to \theta, \psi \to \theta} \proves \theta\)\hfill(\hypertarget{ded:disjElim}{析取消去})
    \begin{proof}
      \leavevmode
      \begin{ND}[][][][\rwidth{ded:conjAssoc2}]
        \ndl{}{\(\phi \lor \psi\)}{假设}\label{1}
        \ndl{}{\(\phi \to \theta\)}{假设}\label{2}
        \ndl{}{\(\psi \to \theta\)}{假设}\label{3}
        \ndl{}{\(\neg\phi \to \psi\)}{\ref{1},定义2.2.4}\label{4}
        \ndl{}{\(\neg\phi \to \theta\)}{\ref{4},\ref{3},示例2.4.19}\label{5}
        \ndl{}{\(\neg\theta \to \phi\)}{\ref{5},\hyperlink{ded:contrapose1}{假言易位1}}\label{6}
        \ndl{}{\((\neg\theta \to \phi) \to (\neg\theta \to \neg\phi) \to \theta\)}{\(\rm P_3\)}\label{7}
        \ndl{}{\(\neg\theta \to \neg\phi\)}{\ref{2},\hyperlink{ded:contrapose1}{假言易位1}}\label{8}
        \ndl{}{\((\neg\theta \to \neg\phi) \to \theta\)}{\ref{6},\ref{7},MP}\label{9}
        \ndl{}{\(\theta\)}{\ref{8},\ref{9},MP}
      \end{ND}
    \end{proof}
  \item \(\Set{\phi \to \psi, \psi \to \phi} \proves \phi \leftrightarrow \psi\)\hfill(等值引入)
    \begin{proof}
      \leavevmode
      \begin{ND}[][][][\rwidth{ded:conjAssoc2}]
        \ndl{}{\(\phi \to \psi\)}{假设}\label{1}
        \ndl{}{\(\psi \to \phi\)}{假设}\label{2}
        \ndl{}{\((\phi \to \psi) \land (\psi \to \phi)\)}{%
          \ref{1},\ref{2}, \hyperlink{ded:conjIntro}{合取引入}}\label{3}
        \ndl{}{\(\phi \leftrightarrow \psi\)}{\ref{3},定义2.2.4}
      \end{ND}
    \end{proof}
    \(\phi \leftrightarrow \psi \proves \phi \to \psi\)\hfill(等值消去)
    \begin{proof}
      \leavevmode
      \begin{ND}[][][][\rwidth{ded:conjAssoc2}]
        \ndl{}{\(\phi \leftrightarrow \psi\)}{假设}\label{1}
        \ndl{}{\((\phi \to \psi) \land (\psi \to \phi)\)}{\ref{1},定义2.2.4}\label{2}
        \ndl{}{\(\phi \to \psi\)}{\ref{2},\hyperlink{ded:conjElim}{合取消去}}
      \end{ND}
    \end{proof}
  \item \(\Set{\neg\phi \to \psi, \neg\phi \to \neg\psi} \proves \phi\)\hfill(反证法)
    \begin{proof}
      \leavevmode
      \begin{ND}[][][][\rwidth{ded:conjAssoc2}]
        \ndl{}{\(\neg\phi \to \psi\)}{假设}\label{1}
        \ndl{}{\(\neg\phi \to \neg\psi\)}{假设}\label{2}
        \ndl{}{\((\neg\phi \to \psi) \to (\neg\phi \to \neg\psi) \to \phi\)}{\(\rm P_3\)}\label{3}
        \ndl{}{\((\neg\phi \to \neg\psi) \to \phi\)}{\ref{1},\ref{3},MP}\label{4}
        \ndl{}{\(\phi\)}{\ref{2},\ref{4},MP}
      \end{ND}
    \end{proof}
    \(\Set{\phi \to \psi, \phi \to \neg\psi} \proves \neg\phi\)\hfill(归谬法)
    \begin{proof}
      \leavevmode
      \begin{ND}[][][][\rwidth{ded:conjAssoc2}]
        \ndl{}{\(\phi \to \psi\)}{假设}\label{1}
        \ndl{}{\(\phi \to \neg\psi\)}{假设}\label{2}
        \ndl{}{\((\phi \to \psi) \to (\phi \to \neg\psi) \to \neg\phi\)}{\hyperref[lem:p3var]{\(\rm P_3\)变体}}\label{3}
        \ndl{}{\((\phi \to \neg\psi) \to \neg\phi\)}{\ref{1},\ref{3},MP}\label{4}
        \ndl{}{\(\neg\phi\)}{\ref{2},\ref{4},MP}
      \end{ND}
      % 或者
      % \begin{ND}[][ded:raa]
      %   \ndl{}{\(\phi \to \psi\)}{假设}\label{1}
      %   \ndl{}{\(\phi \to \neg\psi\)}{假设}\label{2}
      %   \ndl{}{\(\neg\neg\phi \to \phi\)}{\hyperref[lem:dnElim]{双重否定消去引理}}\label{3}
      %   \ndl{}{\(\neg\neg\phi \to \psi\)}{\ref{3},\ref{1},示例2.4.19}\label{4}
      %   \ndl{}{\((\neg\neg\phi \to \psi) \to (\neg\neg\phi \to \neg\psi) \to \neg\phi\)}{\(\rm P_3\)}\label{5}
      %   \ndl{}{\(\neg\neg\phi \to \neg\psi\)}{\ref{3},\ref{2},示例2.4.19}\label{6}
      %   \ndl{}{\((\neg\neg\phi \to \neg\psi) \to \neg\phi\)}{\ref{4},\ref{5},MP}\label{7}
      %   \ndl{}{\(\neg\phi\)}{\ref{6},\ref{7},MP}
      % \end{ND}
    \end{proof}
  \item 假言三段论见示例2.4.19,析取三段论见下。

    \(\Set{\phi \lor \psi, \neg\phi} \proves \psi\)\hfill(析取三段论)
    \begin{proof}
      \leavevmode
      \begin{ND}[][][][\rwidth{ded:conjAssoc2}]
        \ndl{}{\(\phi \lor \psi\)}{假设}\label{1}
        \ndl{}{\(\neg\phi\)}{假设}\label{2}
        \ndl{}{\(\neg\phi \to \psi\)}{\ref{1},定义2.2.4}\label{3}
        \ndl{}{\(\psi\)}{\ref{2},\ref{3},MP}
      \end{ND}
    \end{proof}
  \item \(\phi \to \theta \proves \phi \land \psi \to \theta\)\hfill(前件强化)
    \begin{proof}
      \leavevmode
      \begin{ND}[][ded:sAnte][][\rwidth{ded:conjAssoc2}]
        \ndl{}{\(\phi \to \theta\)}{假设}\label{1}
        \ndl{}{\(\neg\phi \to (\phi \to \neg\psi)\)}{\hyperref[lem:exFalso]{爆炸原理},\hyperref[ded:comm]{交换演绎}}\label{2}
        \ndl{}{\(\neg(\phi \to \neg\psi) \to (\neg\phi \to \neg(\phi \to \neg\psi))\)}{\(\rm P_1\)}\label{3}
        \ndl{}{\((\neg\phi \to \neg(\phi \to \neg\psi)) \to \phi\)}{\ref{1},\ref{2},MP}\label{4}
        \ndl{}{\(\neg(\phi \to \neg\psi) \to \phi\)}{\ref{3},\ref{4},示例2.4.19}\label{5}
        \ndl{}{\(\neg(\phi \to \neg\psi) \to \theta\)}{\ref{5},\ref{1},示例2.4.19}\label{6}
        \ndl{}{\(\phi \land \psi \to \theta\)}{\ref{6},定义2.2.4}
      \end{ND}
    \end{proof}
    \(\Set{\phi \to \psi, \phi \to \theta} \proves \phi \to \psi \land \theta\)\hfill(\hypertarget{ded:sConseq}{后件强化})
    \begin{proof}
      \leavevmode
      \begin{ND}[][][][\rwidth{ded:conjAssoc2}]
        \ndl{}{\(\phi \to \psi\)}{假设}\label{1}
        \ndl{}{\(\phi \to \theta\)}{假设}\label{2}
        \ndl{}{\(\psi \to \theta \to \psi \land \theta\)}{\hyperref[lem:conjIntro]{合取引入引理}}\label{3}
        \ndl{}{\(\phi \to \theta \to \psi \land \theta\)}{\ref{1},\ref{3},示例2.4.19}\label{4}
        \ndl{}{\(\phi \to \psi \land \theta\)}{\ref{2},\ref{4},\hyperref[ded:condMp]{条件MP}}
      \end{ND}
    \end{proof}
  \item \(\Set{\phi \to \theta, \psi \to \theta} \proves \phi \lor \psi \to \theta\)\hfill(前件弱化)
    \begin{proof}
      \leavevmode
      \begin{ND}[][][][\rwidth{ded:conjAssoc2}]
        \ndl{}{\(\phi \to \theta\)}{假设}\label{1}
        \ndl{}{\(\psi \to \theta\)}{假设}\label{2}
        \ndl{}{\(\neg\theta \to \neg\psi\)}{\ref{2},\hyperlink{ded:contrapose1}{假言易位1}}\label{3}
        \ndl{}{\((\neg\theta \to \neg\psi) \to (\neg\theta \to \psi) \to \theta\)}{\(\rm P_3\),\hyperref[ded:comm]{交换演绎}}\label{4}
        \ndl{}{\((\neg\psi \to \phi) \to (\neg\psi \to \theta)\)}{\ref{1},\hyperref[ded:condRepDist]{双件引入}}\label{5}
        \ndl{}{\((\neg\psi \to \theta) \to (\neg\theta \to \psi)\)}{\hyperref[lem:contrapose2]{假言易位引理2}}\label{6}
        \ndl{}{\((\neg\psi \to \phi) \to (\neg\theta \to \psi)\)}{\ref{5},\ref{6},示例2.4.19}\label{7}
        \ndl{}{\((\neg\theta \to \psi) \to \theta\)}{\ref{3},\ref{4},MP}\label{8}
        \ndl{}{\((\neg\phi \to \psi) \to (\neg\psi \to \phi)\)}{\hyperref[lem:contrapose2]{假言易位引理2}}\label{9}
        \ndl{}{\((\neg\psi \to \phi) \to \theta\)}{\ref{7},\ref{8},示例2.4.19}\label{10}
        \ndl{}{\((\neg\phi \to \psi) \to \theta\)}{\ref{9},\ref{10},示例2.4.19}\label{11}
        \ndl{}{\(\phi \lor \psi \to \theta\)}{\ref{11},定义2.2.4}
      \end{ND}
    \end{proof}
    \(\phi \to \theta \proves \phi \to \psi \lor \theta\)\hfill(后件弱化)
    \begin{proof}
      \leavevmode
      \begin{ND}[][][][\rwidth{ded:conjAssoc2}]
        \ndl{}{\(\phi \to \theta\)}{假设}\label{1}
        \ndl{}{\(\theta \to \neg\psi \to \theta\)}{\(\rm P_1\)}\label{2}
        \ndl{}{\(\phi \to \neg\psi \to \theta\)}{\ref{1},\ref{2},示例2.4.19}\label{3}
        \ndl{}{\(\phi \to \psi \lor \theta\)}{\ref{3},定义2.2.4}
      \end{ND}
    \end{proof}
  \item \(\Set{\phi \to \theta, \psi \to \xi, \phi \lor \psi} \proves \theta \lor \xi\)\hfill(\hypertarget{ded:consDl}{建设性两难推理})
    \begin{proof}
      \leavevmode
      \begin{ND}[][][][\rwidth{ded:conjAssoc2}]
        \ndl{}{\(\phi \to \theta\)}{假设}\label{1}
        \ndl{}{\(\psi \to \xi\)}{假设}\label{2}
        \ndl{}{\(\phi \lor \psi\)}{假设}\label{3}
        \ndl{}{\(\neg\phi \to \psi\)}{\ref{3},定义2.2.4}\label{4}
        \ndl{}{\(\neg\phi \to \xi\)}{\ref{4},\ref{2},示例2.4.19}\label{5}
        \ndl{}{\(\neg\xi \to \phi\)}{\ref{5},\hyperlink{ded:contrapose2}{假言易位2}}\label{6}
        \ndl{}{\(\neg\xi \to \theta\)}{\ref{6},\ref{1},示例2.4.19}\label{7}
        \ndl{}{\(\neg\theta \to \xi\)}{\ref{7},\hyperlink{ded:contrapose2}{假言易位2}}\label{8}
        \ndl{}{\(\theta \lor \xi\)}{\ref{8},定义2.2.4}
      \end{ND}
    \end{proof}
    \(\Set{\phi \to \theta, \psi \to \xi, \neg\theta \lor \neg\xi} \proves \neg\phi \lor \neg\psi\)\hfill(破坏性两难推理)
    \begin{proof}
      \leavevmode
      \begin{ND}[][][][\rwidth{ded:conjAssoc2}]
        \ndl{}{\(\phi \to \theta\)}{假设}\label{1}
        \ndl{}{\(\psi \to \xi\)}{假设}\label{2}
        \ndl{}{\(\neg\theta \lor \neg\xi\)}{假设}\label{3}
        \ndl{}{\(\neg\neg\theta \to \neg\xi\)}{\ref{3},定义2.2.4}\label{4}
        \ndl{}{\(\neg\xi \to \neg\psi\)}{\ref{2},\hyperlink{ded:contrapose1}{假言易位1}}\label{5}
        \ndl{}{\(\neg\theta \to \neg\phi\)}{\ref{1},\hyperlink{ded:contrapose1}{假言易位1}}\label{6}
        \ndl{}{\(\neg\neg\theta \to \neg\psi\)}{\ref{4},\ref{5},示例2.4.19}\label{7}
        \ndl{}{\(\neg\theta \lor \neg\neg\theta\)}{示例2.4.10}\label{8}
        \ndl{}{\(\neg\phi \lor \neg\psi\)}{\ref{6},\ref{7},\ref{8},\hyperlink{ded:consDl}{建设性两难推理}}
      \end{ND}
      或者
      \begin{ND}[][][][\rwidth{ded:conjAssoc2}]
        \ndl{}{\(\phi \to \theta\)}{假设}\label{1}
        \ndl{}{\(\psi \to \xi\)}{假设}\label{2}
        \ndl{}{\(\neg\theta \lor \neg\xi\)}{假设}\label{3}
        \ndl{}{\(\neg\neg\theta \to \neg\xi\)}{\ref{3},定义2.2.4}\label{4}
        \ndl{}{\(\xi \to \neg\theta\)}{\ref{4},\hyperlink{ded:contrapose1}{假言易位1}}\label{5}
        \ndl{}{\(\neg\theta \to \neg\phi\)}{\ref{1},\hyperlink{ded:contrapose1}{假言易位1}}\label{6}
        \ndl{}{\(\xi \to \neg\phi\)}{\ref{5},\ref{6},示例2.4.19}\label{7}
        \ndl{}{\(\psi \to \neg\phi\)}{\ref{2},\ref{7},示例2.4.19}\label{8}
        \ndl{}{\(\neg\neg\phi \to \neg\psi\)}{\ref{8},\hyperlink{ded:contrapose1}{假言易位1}}\label{9}
        \ndl{}{\(\neg\phi \lor \neg\psi\)}{\ref{9},定义2.2.4}
      \end{ND}
    \end{proof}
  \item \(\phi \land \phi \proves \phi\)\hfill(幂等律1)
    \begin{proof}
      \leavevmode
      \begin{ND}[][][][\rwidth{ded:conjAssoc2}]
        \ndl{}{\(\phi \land \phi\)}{假设}\label{1}
        \ndl{}{\(\phi\)}{\ref{1},\hyperlink{ded:conjElim}{合取消去}}
      \end{ND}
    \end{proof}
    \(\phi \lor \phi \proves \phi\)\hfill(幂等律2)
    \begin{proof}
      \leavevmode
      \begin{ND}[][][][\rwidth{ded:conjAssoc2}]
        \ndl{}{\(\phi \lor \phi\)}{假设}\label{1}
        \ndl{}{\(\phi \to \phi\)}{示例2.4.8}\label{2}
        \ndl{}{\(\phi\)}{\ref{1},\ref{2},\ref{2},\hyperlink{ded:disjElim}{析取消去}}
      \end{ND}
      或者
      \begin{ND}[][][][\rwidth{ded:conjAssoc2}]
        \ndl{}{\(\phi \lor \phi\)}{假设}\label{1}
        \ndl{}{\(\neg\phi \to \phi\)}{\ref{1},定义2.2.4}\label{2}
        \ndl{}{\((\neg\phi \to \phi) \to (\neg\phi \to \neg\phi) \to \phi\)}{\(\rm P_3\)}\label{3}
        \ndl{}{\(\neg\phi \to \neg\phi\)}{示例2.4.8}\label{4}
        \ndl{}{\((\neg\phi \to \neg\phi) \to \phi\)}{\ref{2},\ref{3},MP}\label{5}
        \ndl{}{\(\phi\)}{\ref{4},\ref{5},MP}
      \end{ND}
    \end{proof}
  \item \(\phi \land \psi \vdashv \psi \land \phi\)\hfill(\hypertarget{ded:conjComm}{交换律1})
    \begin{proof}
      只需证明\(\phi \land \psi \proves \psi \land \phi\)即可,因为\(\psi \land \phi \proves \phi \land \psi\)具有相同的模式。
      \begin{ND}[][][][\rwidth{ded:conjAssoc2}]
        \ndl{}{\(\phi \land \psi\)}{假设}\label{1}
        \ndl{}{\(\phi \land \psi \to \psi\)}{\hyperref[lem:conjElimLeft]{合取左消去引理}}\label{2}
        \ndl{}{\(\psi\)}{\ref{1},\ref{2},MP}\label{3}
        \ndl{}{\(\phi\)}{\ref{1},\hyperlink{ded:conjElim}{合取消去}}\label{4}
        \ndl{}{\(\psi \land \phi\)}{\ref{3},\ref{4},\hyperlink{ded:conjIntro}{合取引入}}
      \end{ND}
      或者
      \begin{ND}[][][][\rwidth{ded:conjAssoc2}]
        \ndl{}{\(\phi \land \psi\)}{假设}\label{1}
        \ndl{}{\((\psi \land \phi) \to (\psi \land \phi)\)}{示例2.4.8}\label{2}
        \ndl{}{\(\psi \to \phi \to (\psi \land \phi)\)}{\ref{2},\hyperlink{ded:exp}{输出律}}\label{3}
        \ndl{}{\(\phi \to \psi \to (\psi \land \phi)\)}{\ref{3},\hyperref[ded:comm]{交换演绎}}\label{4}
        \ndl{}{\((\phi \land \psi) \to (\psi \land \phi)\)}{\ref{4},\hyperlink{ded:exp}{输出律}}\label{5}
        \ndl{}{\(\psi \land \phi\)}{\ref{1},\ref{5},MP}
      \end{ND}
      或者
      \begin{ND}[][][][\rwidth{ded:conjAssoc2}]
        \ndl{}{\(\phi \land \psi\)}{假设}\label{1}
        \ndl{}{\(\neg(\phi \to \neg\psi)\)}{\ref{1},定义2.2.4}\label{2}
        \ndl{}{\((\psi \to \neg\phi) \to (\phi \to \neg\psi)\)}{\hyperref[lem:contrapose3]{假言易位引理3}}\label{3}
        \ndl{}{\(
          \begin{aligned}[t]
            &((\psi \to \neg\phi) \to (\phi \to \neg\psi))
              \to ((\psi \to \neg\phi) \to \neg(\phi \to \neg\psi)) \\
            &\quad\to \neg(\psi \to \neg\phi)
          \end{aligned}
          \)}{\hyperref[lem:p3var]{\(\rm P_3\)变体}}\label{4}
        \ndl{}{\((\psi \to \neg\phi) \to \neg(\phi \to \neg\psi)\)}{\ref{2},示例2.4.18}\label{5}
        \ndl{}{\(((\psi \to \neg\phi) \to \neg(\phi \to \neg\psi))
          \to \neg(\psi \to \neg\phi)\)}{\ref{3},\ref{4},MP}\label{6}
        \ndl{}{\(\neg(\psi \to \neg\phi)\)}{\ref{5},\ref{6},MP}\label{7}
        \ndl{}{\(\psi \land \phi\)}{\ref{7},定义2.2.4}
      \end{ND}
    \end{proof}
    \(\phi \lor \psi \vdashv \psi \lor \phi\)\hfill(\hypertarget{ded:disjComm}{交换律2})
    \begin{proof}
      同上,只需证明\(\phi \lor \psi \proves \psi \lor \phi\)即可。
      \begin{ND}[][][][\rwidth{ded:conjAssoc2}]
        \ndl{}{\(\phi \lor \psi\)}{假设}\label{1}
        \ndl{}{\(\neg\phi \to \psi\)}{\ref{1},定义2.2.4}\label{2}
        \ndl{}{\(\neg\psi \to \phi\)}{\ref{2},\hyperlink{ded:contrapose2}{假言易位2}}\label{3}
        \ndl{}{\(\psi \lor \phi\)}{\ref{3},定义2.2.4}
      \end{ND}
    \end{proof}
  \item \(\neg(\phi \land \psi) \vdashv \neg\phi \lor \neg\psi\)\hfill(德·摩根律1)
    \begin{proof}
      先证\(\neg(\phi \land \psi) \proves \neg\phi \lor \neg\psi\)。
      \begin{ND}[][][][\rwidth{ded:conjAssoc2}]
        \ndl{}{\(\neg(\phi \land \psi)\)}{假设}\label{1}
        \ndl{}{\(\neg\neg(\phi \to \neg\psi)\)}{\ref{1},定义2.2.4}\label{2}
        \ndl{}{\(\phi \to \neg\psi\)}{\ref{2},\hyperlink{ded:dnElim}{双重否定消去}}\label{3}
        \ndl{}{\(\neg\phi \lor \neg\psi\)}{\ref{3},定义2.2.4}
      \end{ND}
      再证\(\neg\phi \lor \neg\psi \proves \neg(\phi \land \psi)\)。
      \begin{ND}[][][][\rwidth{ded:conjAssoc2}]
        \ndl{}{\(\neg\phi \lor \neg\psi\)}{假设}\label{1}
        \ndl{}{\(\phi \to \neg\psi\)}{\ref{1},定义2.2.4}\label{2}
        \ndl{}{\(\neg\neg(\phi \to \neg\psi)\)}{\ref{2},\hyperlink{ded:dnIntro}{双重否定引入}}\label{3}
        \ndl{}{\(\neg(\phi \land \psi)\)}{\ref{3},定义2.2.4}
      \end{ND}
    \end{proof}
    \(\neg(\phi \lor \psi) \vdashv \neg\phi \land \neg\psi\)\hfill(德·摩根律2)
    \begin{proof}
      先证\(\neg(\phi \lor \psi) \proves \neg\phi \land \neg\psi\)。
      \begin{ND}[][ded:dmDisjElim][][\rwidth{ded:conjAssoc2}]
        \ndl{}{\(\neg(\phi \lor \psi)\)}{假设}\label{1}
        \ndl{}{\(\neg(\phi \lor \psi) \to (\neg\phi \land \neg\psi)\)}{\hyperref[lem:dm1]{德·摩根引理1}}\label{2}
        \ndl{}{\(\neg\phi \land \neg\psi\)}{\ref{1},\ref{2},MP}
      \end{ND}
      再证\(\neg\phi \land \neg\psi \proves \neg(\phi \lor \psi)\)。
      \begin{ND}[][][][\rwidth{ded:conjAssoc2}]
        \ndl{}{\(\neg\phi \land \neg\psi\)}{假设}\label{1}
        \ndl{}{\((\neg\phi \land \neg\psi) \to \neg(\phi \lor \psi)\)}{\hyperref[lem:dm2]{德·摩根引理2}}\label{2}
        \ndl{}{\(\neg(\phi \lor \psi)\)}{\ref{1},\ref{2},MP}
      \end{ND}
    \end{proof}
  \item \(\phi \land (\psi \land \theta) \vdashv (\phi \land \psi) \land \theta\)\hfill(\hypertarget{ded:conjAssoc}{结合律1})
    \begin{proof}
      先证\(\phi \land (\psi \land \theta) \proves (\phi \land \psi) \land \theta\)。
      \begin{ND}[][][][\rwidth{ded:conjAssoc2}]
        \ndl{}{\(\phi \land (\psi \land \theta)\)}{假设}\label{1}
        \ndl{}{\((\psi \land \theta) \land \phi\)}{\ref{1},\hyperlink{ded:conjComm}{交换律1}}\label{2}
        \ndl{}{\(\psi \land \theta\)}{\ref{2},\hyperlink{ded:conjElim}{合取消去}}\label{3}
        \ndl{}{\(\theta \land \psi\)}{\ref{3},\hyperlink{ded:conjComm}{交换律1}}\label{4}
        \ndl{}{\(\phi\)}{\ref{1},\hyperlink{ded:conjElim}{合取消去}}\label{5}
        \ndl{}{\(\psi\)}{\ref{3},\hyperlink{ded:conjElim}{合取消去}}\label{6}
        \ndl{}{\(\phi \land \psi\)}{\ref{5},\ref{6},\hyperlink{ded:conjIntro}{合取引入}}\label{7}
        \ndl{}{\(\theta\)}{\ref{4},\hyperlink{ded:conjElim}{合取消去}}\label{8}
        \ndl{}{\((\phi \land \psi) \land \theta\)}{\ref{7},\ref{8},\hyperlink{ded:conjIntro}{合取引入}}
      \end{ND}
      再证\((\phi \land \psi) \land \theta \proves \phi \land (\psi \land \theta)\)。
      \begin{ND}[][ded:conjAssoc2]
        \ndl{}{\((\phi \land \psi) \land \theta\)}{假设}\label{1}
        \ndl{}{\(\theta \land (\phi \land \psi)\)}{\ref{1},\hyperlink{ded:conjComm}{交换律1}}\label{2}
        \ndl{}{\((\theta \land \phi) \land \psi\)}{\ref{2},\hyperlink{ded:conjAssoc}{此证明第1部分}}\label{3}
        \ndl{}{\(\psi \land (\theta \land \phi)\)}{\ref{3},\hyperlink{ded:conjComm}{交换律1}}\label{4}
        \ndl{}{\((\psi \land \theta) \land \phi\)}{\ref{4},\hyperlink{ded:conjAssoc}{此证明第1部分}}\label{5}
        \ndl{}{\(\phi \land (\psi \land \theta)\)}{\ref{5},\hyperlink{ded:conjComm}{交换律1}}
      \end{ND}
      或者仿照此证明的第1部分,思路完全一样。
      % \begin{ND}[][][][\rwidth{ded:conjAssoc2}]
      %   \ndl{}{\((\phi \land \psi) \land \theta\)}{假设}\label{1}
      %   \ndl{}{\(\theta \land (\phi \land \psi)\)}{\ref{1},\hyperlink{ded:conjComm}{交换律1}}\label{2}
      %   \ndl{}{\(\phi \land \psi\)}{\ref{1},\hyperlink{ded:conjElim}{合取消去}}\label{3}
      %   \ndl{}{\(\psi \land \phi\)}{\ref{3},\hyperlink{ded:conjComm}{交换律1}}\label{4}
      %   \ndl{}{\(\psi\)}{\ref{4},\hyperlink{ded:conjElim}{合取消去}}\label{5}
      %   \ndl{}{\(\theta\)}{\ref{2},\hyperlink{ded:conjElim}{合取消去}}\label{6}
      %   \ndl{}{\(\phi\)}{\ref{3},\hyperlink{ded:conjElim}{合取消去}}\label{7}
      %   \ndl{}{\(\psi \land \theta\)}{\ref{5},\ref{6},\hyperlink{ded:conjIntro}{合取引入}}\label{8}
      %   \ndl{}{\(\phi \land (\psi \land \theta)\)}{\ref{7},\ref{8},\hyperlink{ded:conjIntro}{合取引入}}
      % \end{ND}
    \end{proof}
    \(\phi \lor (\psi \lor \theta) \vdashv (\phi \lor \psi) \lor \theta\)\hfill(\hypertarget{ded:disjAssoc}{结合律2})
    \begin{proof}
      先证\(\phi \lor (\psi \lor \theta) \proves (\phi \lor \psi) \lor \theta\)。
      \begin{ND}[][][][\rwidth{ded:conjAssoc2}]
        \ndl{}{\(\phi \lor (\psi \lor \theta)\)}{假设}\label{1}
        \ndl{}{\(\neg\phi \to (\neg\psi \to \theta)\)}{\ref{1},定义2.2.4}\label{2}
        \ndl{}{\(\neg(\phi \lor \psi) \to (\neg\phi \land \neg\psi)\)}{\hyperref[lem:dm1]{德·摩根引理1}}\label{3}
        \ndl{}{\((\neg\phi \land \neg\psi) \to \theta\)}{\ref{2},\hyperlink{ded:exp}{输出律}}\label{4}
        \ndl{}{\(\neg(\phi \lor \psi) \to \theta\)}{\ref{3},\ref{4},示例2.4.19}\label{5}
        \ndl{}{\((\phi \lor \psi) \lor \theta\)}{\ref{5},定义2.2.4}
      \end{ND}
      再证\((\phi \lor \psi) \lor \theta \proves \phi \lor (\psi \lor \theta)\)。
      \begin{ND}[][][][\rwidth{ded:conjAssoc2}]
        \ndl{}{\((\phi \lor \psi) \lor \theta\)}{假设}\label{1}
        \ndl{}{\(\theta \lor (\phi \lor \psi)\)}{\ref{1},\hyperlink{ded:disjComm}{交换律2}}\label{2}
        \ndl{}{\((\theta \lor \phi) \lor \psi\)}{\ref{2},\hyperlink{ded:disjAssoc}{此证明第1部分}}\label{3}
        \ndl{}{\(\psi \lor (\theta \lor \phi)\)}{\ref{3},\hyperlink{ded:disjComm}{交换律2}}\label{4}
        \ndl{}{\((\psi \lor \theta) \lor \phi\)}{\ref{4},\hyperlink{ded:disjAssoc}{此证明第1部分}}\label{5}
        \ndl{}{\(\phi \lor (\psi \lor \theta)\)}{\ref{5},\hyperlink{ded:disjComm}{交换律2}}
      \end{ND}
      或者
      \begin{ND}[][][][\rwidth{ded:conjAssoc2}]
        \ndl{}{\((\phi \lor \psi) \lor \theta\)}{假设}\label{1}
        \ndl{}{\((\neg\phi \land \neg\psi) \to \neg(\phi \lor \psi)\)}{\hyperref[lem:dm2]{德·摩根引理2}}\label{2}
        \ndl{}{\(\neg(\phi \lor \psi) \to \theta\)}{\ref{1},定义2.2.4}\label{3}
        \ndl{}{\((\neg\phi \land \neg\psi) \to \theta\)}{\ref{2},\ref{3},示例2.4.19}\label{4}
        \ndl{}{\(\neg\phi \to (\neg\psi \to \theta)\)}{\ref{4},\hyperlink{ded:exp}{输出律}}\label{5}
        \ndl{}{\(\phi \lor (\psi \lor \theta)\)}{\ref{5},定义2.2.4}
      \end{ND}
    \end{proof}
  \item \(\phi \land (\psi \lor \theta) \vdashv (\phi \land \psi) \lor (\phi \land \theta)\)\hfill(分配律1)
    \begin{proof}
      先证\(\phi \land (\psi \lor \theta) \proves (\phi \land \psi) \lor (\phi \land \theta)\)。
      \begin{ND}[][][][\rwidth{ded:conjAssoc2}]
        \ndl{}{\(\phi \land (\psi \lor \theta)\)}{假设}\label{1}
        \ndl{}{\((\psi \lor \theta) \land \phi\)}{\ref{1},\hyperlink{ded:conjComm}{交换律1}}\label{2}
        \ndl{}{\(\phi\)}{\ref{1},\hyperlink{ded:conjElim}{合取消去}}\label{3}
        \ndl{}{\(\phi \to \psi \to \phi \land \psi\)}{\hyperref[lem:conjIntro]{合取引入引理}}\label{4}
        \ndl{}{\(\phi \to \theta \to \phi \land \theta\)}{\hyperref[lem:conjIntro]{合取引入引理}}\label{5}
        \ndl{}{\(\psi \to \phi \land \psi\)}{\ref{3},\ref{4},MP}\label{6}
        \ndl{}{\(\theta \to \phi \land \theta\)}{\ref{3},\ref{5},MP}\label{7}
        \ndl{}{\(\psi \lor \theta\)}{\ref{2},\hyperlink{ded:conjElim}{合取消去}}\label{8}
        \ndl{}{\((\phi \land \psi) \lor (\phi \land \theta)\)}{\ref{6},\ref{7},\ref{8},\hyperlink{ded:consDl}{建设性两难推理}}
      \end{ND}
      再证\((\phi \land \psi) \lor (\phi \land \theta) \proves \phi \land (\psi \lor \theta)\)。
      \begin{ND}[][][][\rwidth{ded:conjAssoc2}]
        \ndl{}{\((\phi \land \psi) \lor (\phi \land \theta)\)}{假设}\label{1}
        \ndl{}{\(\phi \land \psi \to \psi\)}{\hyperref[lem:conjElimLeft]{合取左消去引理}}\label{2}
        \ndl{}{\(\psi \to (\neg\psi \to \theta)\)}{\hyperref[lem:exFalso]{爆炸原理}}\label{3}
        \ndl{}{\(\phi \land \psi \to \phi\)}{\hyperref[lem:conjElimRight]{合取右消去引理}}\label{4}
        \ndl{}{\(\phi \land \psi \to (\neg\psi \to \theta)\)}{\ref{2},\ref{3},示例2.4.19}\label{5}
        \ndl{}{\(\phi \land \theta \to \theta\)}{\hyperref[lem:conjElimLeft]{合取左消去引理}}\label{6}
        \ndl{}{\(\theta \to (\neg\psi \to \theta)\)}{\(\rm P_1\)}\label{7}
        \ndl{}{\(\phi \land \theta \to \phi\)}{\hyperref[lem:conjElimRight]{合取右消去引理}}\label{8}
        \ndl{}{\(\phi \land \theta \to (\neg\psi \to \theta)\)}{\ref{6},\ref{7},示例2.4.19}\label{9}
        \ndl{}{\(\phi \land \psi \to \phi \land (\neg\psi \to \theta)\)}{\ref{4},\ref{5},\hyperlink{ded:sConseq}{后件强化}}\label{10}
        \ndl{}{\(\phi \land \theta \to \phi \land (\neg\psi \to \theta)\)}{\ref{8},\ref{9},\hyperlink{ded:sConseq}{后件强化}}\label{11}
        \ndl{}{\(\phi \land (\neg\psi \to \theta)\)}{\ref{10},\ref{11},\ref{1},\hyperlink{ded:consDl}{建设性两难推理}}\label{12}
        \ndl{}{\(\phi \land (\psi \lor \theta)\)}{\ref{12},定义2.2.4}
      \end{ND}
    \end{proof}
    \(\phi \lor (\psi \land \theta) \vdashv (\phi \lor \psi) \land (\phi \lor \theta)\)\hfill(分配律2)
    \begin{proof}
      先证\(\phi \lor (\psi \land \theta) \proves (\phi \lor \psi) \land (\phi \lor \theta)\)。
      \begin{ND}[][][][\rwidth{ded:conjAssoc2}]
        \ndl{}{\(\phi \lor (\psi \land \theta)\)}{假设}\label{1}
        \ndl{}{\(\neg\phi \to (\psi \land \theta)\)}{\ref{1},定义2.2.4}\label{2}
        \ndl{}{\((\psi \land \theta) \to \psi\)}{\hyperref[lem:conjElimRight]{合取右消去引理}}\label{3}
        \ndl{}{\((\psi \land \theta) \to \theta\)}{\hyperref[lem:conjElimLeft]{合取左消去引理}}\label{4}
        \ndl{}{\(\neg\phi \to \psi\)}{\ref{2},\ref{3},示例2.4.19}\label{5}
        \ndl{}{\(\neg\phi \to \theta\)}{\ref{2},\ref{4},示例2.4.19}\label{6}
        \ndl{}{\((\neg\phi \to \psi) \land (\neg\phi \to \theta)\)}{\ref{5},\ref{6},\hyperlink{ded:conjIntro}{合取引入}}\label{7}
        \ndl{}{\((\phi \lor \psi) \land (\phi \lor \theta)\)}{\ref{7},定义2.2.4}
      \end{ND}
      或者
      \begin{ND}[][][][\rwidth{ded:conjAssoc2}]
        \ndl{}{\(\phi \lor (\psi \land \theta)\)}{假设}\label{1}
        \ndl{}{\(\phi \to (\neg\phi \to \psi)\)}{\hyperref[lem:exFalso]{爆炸原理}}\label{2}
        \ndl{}{\(\phi \to (\neg\phi \to \theta)\)}{\hyperref[lem:exFalso]{爆炸原理}}\label{3}
        \ndl{}{\(\psi \land \theta \to \psi\)}{\hyperref[lem:conjElimRight]{合取右消去引理}}\label{4}
        \ndl{}{\(\psi \to (\neg\phi \to \psi)\)}{\(\rm P_1\)}\label{5}
        \ndl{}{\(\psi \land \theta \to \theta\)}{\hyperref[lem:conjElimLeft]{合取左消去引理}}\label{6}
        \ndl{}{\(\theta \to (\neg\phi \to \theta)\)}{\(\rm P_1\)}\label{7}
        \ndl{}{\(\psi \land \theta \to (\neg\phi \to \psi)\)}{\ref{4},\ref{5},示例2.4.19}\label{8}
        \ndl{}{\(\psi \land \theta \to (\neg\phi \to \theta)\)}{\ref{6},\ref{7},示例2.4.19}\label{9}
        \ndl{}{\(\phi \to (\neg\phi \to \psi) \land (\neg\phi \to \theta)\)}{\ref{2},\ref{3},\hyperlink{ded:sConseq}{后件强化}}\label{10}
        \ndl{}{\(\psi \land \theta \to (\neg\phi \to \psi) \land (\neg\phi \to \theta)\)}{\ref{8},\ref{9},\hyperlink{ded:sConseq}{后件强化}}\label{11}
        \ndl{}{\((\neg\phi \to \psi) \land (\neg\phi \to \theta)\)}{\ref{10},\ref{11},\ref{1},\hyperlink{ded:consDl}{建设性两难推理}}\label{12}
        \ndl{}{\((\phi \lor \psi) \land (\phi \lor \theta)\)}{\ref{12},定义2.2.4}
      \end{ND}
      再证\((\phi \lor \psi) \land (\phi \lor \theta) \proves \phi \lor (\psi \land \theta)\)。
      \begin{ND}[][][][\rwidth{ded:conjAssoc2}]
        \ndl{}{\((\phi \lor \psi) \land (\phi \lor \theta)\)}{假设}\label{1}
        \ndl{}{\((\neg\phi \to \psi) \land (\neg\phi \to \theta)\)}{\ref{1},定义2.2.4}\label{2}
        \ndl{}{\((\neg\phi \to \theta) \land (\neg\phi \to \psi)\)}{\ref{2},\hyperlink{ded:conjComm}{交换律1}}\label{3}
        \ndl{}{\(\neg\phi \to \psi\)}{\ref{2},\hyperlink{ded:conjElim}{合取消去}}\label{4}
        \ndl{}{\(\neg\phi \to \theta\)}{\ref{3},\hyperlink{ded:conjElim}{合取消去}}\label{5}
        \ndl{}{\(\neg\phi \to (\psi \land \theta)\)}{\ref{4},\ref{5},\hyperlink{ded:sConseq}{后件强化}}\label{6}
        \ndl{}{\(\phi \lor (\psi \land \theta)\)}{\ref{6},定义2.2.4}
      \end{ND}
    \end{proof}
  \item \(\phi \to \psi \vdashv \phi \to \phi \land \psi\)\hfill(吸收律)
    \begin{proof}
      先证\(\phi \to \psi \proves \phi \to \phi \land \psi\)。
      \begin{ND}[][][][\rwidth{ded:conjAssoc2}]
        \ndl{}{\(\phi \to \psi\)}{假设}\label{1}
        \ndl{}{\(\phi \to \phi\)}{示例2.4.8}\label{2}
        \ndl{}{\(\phi \to \phi \land \psi\)}{\ref{2},\ref{1},\hyperlink{ded:sConseq}{后件强化}}
      \end{ND}
      再证\(\phi \to \phi \land \psi \proves \phi \to \psi\)。
      \begin{ND}[][][][\rwidth{ded:conjAssoc2}]
        \ndl{}{\(\phi \to \phi \land \psi\)}{假设}\label{1}
        \ndl{}{\(\phi \land \psi \to \psi\)}{\hyperref[lem:conjElimLeft]{合取左消去引理}}\label{2}
        \ndl{}{\(\phi \to \psi\)}{\ref{1},\ref{2},示例2.4.19}
      \end{ND}
      \begin{remark}
        吸收律(law of absorption)见于《数学原理》第1卷\href{https://hdl.handle.net/2027/miun.aat3201.0001.001?urlappend=%3Bseq=36}{第14页}。根据此书,该律之所以叫作“吸收律”,是因为合取式中的因子\(\psi\)被因子\(\phi\)吸收掉了,如果\(\phi\)蕴涵\(\psi\)的话。实际上,原始形式的公式用今天的符号来表达,大约可以写成
        \begin{equation*}
          \asserts
          (\phi \to \psi)
          \leftrightarrow
          (\phi \leftrightarrow \phi \land \psi)。
        \end{equation*}
        注意到其中第二个等值符号,这样前面解释“吸收”的说法就容易理解了。
      \end{remark}
    \end{proof}
    \(\phi \to \psi \to \theta \vdashv \phi \land \psi \to \theta\)\hfill(\hypertarget{ded:exp}{输出律})
    \begin{proof}
      先证\(\phi \to \psi \to \theta \proves \phi \land \psi \to \theta\)。
      \begin{ND}[][][][\rwidth{ded:conjAssoc2}]
        \ndl{}{\(\phi \to \psi \to \theta\)}{假设}\label{1}
        \ndl{}{\(\phi \land \psi \to \phi\)}{\hyperref[lem:conjElimRight]{合取右消去引理}}\label{2}
        \ndl{}{\(\phi \land \psi \to \psi\)}{\hyperref[lem:conjElimLeft]{合取左消去引理}}\label{3}
        \ndl{}{\(\phi \land \psi \to \psi \to \theta\)}{\ref{2},\ref{1},示例2.4.19}\label{4}
        \ndl{}{\(\phi \land \psi \to \theta\)}{\ref{3},\ref{4},\hyperref[ded:condMp]{条件MP}}
      \end{ND}
      再证\(\phi \land \psi \to \theta \proves \phi \to \psi \to \theta\)。
      \begin{ND}[][][][\rwidth{ded:conjAssoc2}]
        \ndl{}{\(\phi \land \psi \to \theta\)}{假设}\label{1}
        \ndl{}{\(\phi \to (\psi \to \phi \land \psi)\)}{\hyperref[lem:conjIntro]{合取引入引理}}\label{2}
        \ndl{}{\((\psi \to \phi \land \psi) \to (\psi \to \theta)\)}{\ref{1},\hyperref[ded:condRepDist]{双件引入}}\label{3}
        \ndl{}{\(\phi \to \psi \to \theta\)}{\ref{2},\ref{3},示例2.4.19}
      \end{ND}
      \begin{remark}
        一些逻辑学家会将该律称作Curry公式,虽然此公式早在1894年就见于Cesare Burali-Forti的著作《数理逻辑》\href{https://archive.org/details/BuraliFortiLogicaMatematica/page/n28/mode/1up}{第24页}。“输出律”的名字是Peano在《数理逻辑研究》(1897)一文中的\href{https://archive.org/details/peano-studii-di-logica-matematica/page/18/mode/1up}{第18页}给出的,但仅仅是指\(\dashv\)的方向。他在同年出版的《数学的形式》第2版中的\href{https://gallica.bnf.fr/ark:/12148/bpt6k84142s/f39.item}{第38页}把\(\vdash\)的方向称作“输入律”。
      \end{remark}
    \end{proof}
  \item \(\phi \to \psi \vdashv \neg\psi \to \neg\phi\)\hfill(\hypertarget{ded:contrapose1}{假言易位1})
    \begin{proof}
      先证\(\neg\psi \to \neg\phi \proves \phi \to \psi\),如下。
      \begin{ND}[][][][\rwidth{ded:conjAssoc2}]
        \ndl{}{\(\neg\psi \to \neg\phi\)}{假设}\label{1}
        \ndl{}{\((\neg\psi \to \neg\phi) \to (\phi \to \psi)\)}{\hyperref[lem:contrapose1]{假言易位引理1}}\label{2}
        \ndl{}{\(\phi \to \psi\)}{\ref{1},\ref{2},MP}
      \end{ND}
      再证\(\phi \to \psi \proves \neg\psi \to \neg\phi\),如下。
      \begin{ND}[][][][\rwidth{ded:conjAssoc2}]
        \ndl{}{\(\phi \to \psi\)}{假设}\label{1}
        \ndl{}{\((\phi \to \psi) \to (\phi \to \neg\psi) \to \neg\phi\)}{\hyperref[lem:p3var]{\(\rm P_3\)变体}}\label{2}
        \ndl{}{\(\neg\psi \to (\phi \to \neg\psi)\)}{\(\rm P_1\)}\label{3}
        \ndl{}{\((\phi \to \neg\psi) \to \neg\phi\)}{\ref{1},\ref{2},MP}\label{4}
        \ndl{}{\(\neg\psi \to \neg\phi\)}{\ref{3},\ref{4},示例2.4.19}
      \end{ND}
    \end{proof}
    \(\neg\phi \to \psi \vdashv \neg\psi \to \phi\)\hfill(\hypertarget{ded:contrapose2}{假言易位2})
    \begin{proof}
      只需证明\(\neg\phi \to \psi \proves \neg\psi \to \phi\)即可,因为\(\neg\psi \to \phi \proves \neg\phi \to \psi\)和前者是同一种模式。
      \begin{ND}[][][][\rwidth{ded:conjAssoc2}]
        \ndl{}{\(\neg\phi \to \psi\)}{假设}\label{1}
        \ndl{}{\((\neg\phi \to \psi) \to (\neg\psi \to \phi)\)}{\hyperref[lem:contrapose2]{假言易位引理2}}\label{2}
        \ndl{}{\(\neg\psi \to \phi\)}{\ref{1},\ref{2},MP}
      \end{ND}
      % 或者
      % \begin{ND}[][][][\rwidth{ded:conjAssoc2}]
      %   \ndl{}{\(\neg\phi \to \psi\)}{假设}\label{1}
      %   \ndl{}{\((\neg\phi \to \psi) \to (\neg\phi \to \neg\psi) \to \phi\)}{\(\rm P_3\)}\label{2}
      %   \ndl{}{\(\neg\psi \to (\neg\phi \to \neg\psi)\)}{\(\rm P_1\)}\label{3}
      %   \ndl{}{\((\neg\phi \to \neg\psi) \to \phi\)}{\ref{1},\ref{2},MP}\label{4}
      %   \ndl{}{\(\neg\psi \to \phi\)}{\ref{3},\ref{4},示例2.4.19}
      % \end{ND}
    \end{proof}
    \(\phi \to \neg\psi \vdashv \psi \to \neg\phi\)\hfill(假言易位3)
    \begin{proof}
      同上,只需证明\(\phi \to \neg\psi \proves \psi \to \neg\phi\)。
      \begin{ND}[][][][\rwidth{ded:conjAssoc2}]
        \ndl{}{\(\phi \to \neg\psi\)}{假设}\label{1}
        \ndl{}{\((\phi \to \neg\psi) \to (\psi \to \neg\phi)\)}{\hyperref[lem:contrapose3]{假言易位引理3}}\label{2}
        \ndl{}{\(\psi \to \neg\phi\)}{\ref{1},\ref{2},MP}
      \end{ND}
      或者
      \begin{ND}[][][][\rwidth{ded:conjAssoc2}]
        \ndl{}{\(\phi \to \neg\psi\)}{假设}\label{1}
        \ndl{}{\((\phi \to \neg\psi) \to (\phi \to \psi) \to \neg\phi\)}{\hyperref[lem:p3var]{\(\rm P_3\)变体},\hyperref[ded:comm]{交换演绎}}\label{2}
        \ndl{}{\(\psi \to (\phi \to \psi)\)}{\(\rm P_1\)}\label{3}
        \ndl{}{\((\phi \to \psi) \to \neg\phi\)}{\ref{1},\ref{2},MP}\label{4}
        \ndl{}{\(\psi \to \neg\phi\)}{\ref{3},\ref{4},示例2.4.19}
      \end{ND}
      % 或者
      % \begin{ND}[][][][\rwidth{ded:conjAssoc2}]
      %   \ndl{}{\(\phi \to \neg\psi\)}{假设}\label{1}
      %   \ndl{}{\((\phi \to \neg\psi) \to (\neg\neg\phi \to \neg\psi)\)}{\hyperref[lem:dnIntroAnte]{前件双重否定引理}}\label{2}
      %   \ndl{}{\(\neg\neg\phi \to \neg\psi\)}{\ref{1},\ref{2},MP}\label{3}
      %   \ndl{}{\(\psi \to \neg\phi\)}{\ref{3},\hyperlink{ded:contrapose1}{假言易位1}}
      % \end{ND}
    \end{proof}
  \item \(\phi \leftrightarrow \psi \vdashv \neg\phi \leftrightarrow \neg\psi\)\hfill(等值否定交换1)
    \begin{proof}
      先证\(\phi \leftrightarrow \psi \proves \neg\phi \leftrightarrow \neg\psi\)。
      \begin{ND}[][ded:equiv][][\rwidth{ded:conjAssoc2}]
        \ndl{}{\(\phi \leftrightarrow \psi\)}{假设}\label{1}
        \ndl{}{\((\phi \to \psi) \land (\psi \to \phi)\)}{\ref{1},定义2.2.4}\label{2}
        \ndl{}{\((\psi \to \phi) \land (\phi \to \psi)\)}{\ref{2},\hyperlink{ded:conjComm}{交换律1}}\label{3}
        \ndl{}{\(\phi \to \psi\)}{\ref{2},\hyperlink{ded:conjElim}{合取消去}}\label{4}
        \ndl{}{\(\psi \to \phi\)}{\ref{3},\hyperlink{ded:conjElim}{合取消去}}\label{5}
        \ndl{}{\(\neg\phi \to \neg\psi\)}{\ref{5},\hyperlink{ded:contrapose1}{假言易位1}}\label{6}
        \ndl{}{\(\neg\psi \to \neg\phi\)}{\ref{6},\hyperlink{ded:contrapose1}{假言易位1}}\label{7}
        \ndl{}{\((\neg\phi \to \neg\psi) \land (\neg\psi \to \neg\phi)\)}{\ref{6},\ref{7},\hyperlink{ded:conjIntro}{合取引入}}\label{8}
        \ndl{}{\(\neg\phi \leftrightarrow \neg\psi\)}{\ref{8},定义2.2.4}
      \end{ND}
      再证\(\neg\phi \leftrightarrow \neg\psi \proves \phi \leftrightarrow \psi\)。可以仿照上面的演绎,步骤完全一样,就是在使用假言易位1的时候,使用的方向不一样。下面是另外一种思路。
      \begin{ND}[][][][\rwidth{ded:conjAssoc2}]
        \ndl{}{\(\neg\phi \leftrightarrow \neg\psi\)}{假设}\label{1}
        \ndl{}{\(\neg\neg\phi \leftrightarrow \neg\neg\psi\)}{\ref{1},此证明第1部分}\label{2}
        \ndl{}{\((\neg\neg\phi \to \neg\neg\psi) \land (\neg\neg\psi \to \neg\neg\phi)\)}{\ref{2},定义2.2.4}\label{3}
        \ndl{}{\((\neg\neg\psi \to \neg\neg\phi) \land (\neg\neg\phi \to \neg\neg\psi)\)}{\ref{3},\hyperlink{ded:conjComm}{交换律1}}\label{4}
        \ndl{}{\(\neg\neg\phi \to \neg\neg\psi\)}{\ref{3},\hyperlink{ded:conjElim}{合取消去}}\label{5}
        \ndl{}{\(\neg\neg\psi \to \neg\neg\phi\)}{\ref{4},\hyperlink{ded:conjElim}{合取消去}}\label{6}
        \ndl{}{\(\phi \to \neg\neg\phi\)}{\hyperref[lem:dnIntro]{双重否定引入引理}}\label{7}
        \ndl{}{\(\neg\neg\psi \to \psi\)}{\hyperref[lem:dnElim]{双重否定消去引理}}\label{8}
        \ndl{}{\(\psi \to \neg\neg\psi\)}{\hyperref[lem:dnIntro]{双重否定引入引理}}\label{9}
        \ndl{}{\(\neg\neg\phi \to \phi\)}{\hyperref[lem:dnElim]{双重否定消去引理}}\label{10}
        \ndl{}{\(\phi \to \neg\neg\psi\)}{\ref{7},\ref{5},示例2.4.19}\label{11}
        \ndl{}{\(\phi \to \psi\)}{\ref{11},\ref{8},示例2.4.19}\label{12}
        \ndl{}{\(\psi \to \neg\neg\phi\)}{\ref{9},\ref{6},示例2.4.19}\label{13}
        \ndl{}{\(\psi \to \phi\)}{\ref{13},\ref{10},示例2.4.19}\label{14}
        \ndl{}{\((\phi \to \psi) \land (\psi \to \phi)\)}{\ref{13},\ref{14},\hyperlink{ded:conjIntro}{合取引入}}\label{15}
        \ndl{}{\(\phi \leftrightarrow \psi\)}{\ref{15},定义2.2.4}
      \end{ND}
    \end{proof}
    \(\neg\phi \leftrightarrow \psi \vdashv \phi \leftrightarrow \neg\psi\)\hfill(等值否定交换2)
    \begin{proof}
      先证\(\neg\phi \leftrightarrow \psi \proves \phi \leftrightarrow \neg\psi\)。只需将前一个证明第1部分第\ndref{ded:equiv}{6}、\ndref{ded:equiv}{7}步中的假言易位1依次改成假言易位3和2即可。

      再证\(\phi \leftrightarrow \neg\psi \proves \neg\phi \leftrightarrow \psi\)。只需将前一个证明第1部分第\ndref{ded:equiv}{6}、\ndref{ded:equiv}{7}步中的假言易位1依次改成假言易位2和3即可。
    \end{proof}
    % \(\phi \leftrightarrow \neg\psi \proves \neg\phi \leftrightarrow \psi\)\hfill(等值否定交换3)
  \end{enumerate}
\end{description}

\end{document}

% Local Variables:
% TeX-engine: luatex
% End:
