\newcommand*{\titleContent}{《数理逻辑》习题}
\newcommand*{\authorContent}{赵磊}
\newif\ifbsixpaper
% \bsixpapertrue
\ifbsixpaper
\documentclass[punct=custom/kaiming,fontset=none]{ctexart}
\usepackage[b6paper,hmargin=.4in,vmargin=.3in]{geometry}
\title{\Large\bfseries\titleContent}
\else
\documentclass[a4paper,punct=custom/kaiming,fontset=none]{ctexart}
\usepackage[hmargin=1in,vmargin=1in]{geometry}
\title{\bfseries\titleContent}
\fi

\author{\authorContent}

\usepackage{mathtools,amssymb,amsthm}
\newcommand\SetSymbol[1][]{%
  \nonscript\:#1\vert
  \allowbreak
  \nonscript\:
  \mathopen{}}
% \DeclarePairedDelimiterX{\paren}[1]{\lparen}{\rparen}{%
%   \renewcommand{\given}{\SetSymbol[\delimsize]}#1}
\DeclarePairedDelimiterX{\brce}[1]{\lbrace}{\rbrace}{%
  \newcommand{\given}{\SetSymbol[\delimsize]}#1}
\usepackage{lualatex-math}

\newif\ifadobe
% \adobetrue

\ifadobe
\setmainfont{Latin Modern Roman}[BoldFont=Adobe Heiti Std]
\setCJKmainfont{Adobe Song Std}[BoldFont=Adobe Heiti Std]
\else
\setmainfont{Times New Roman}[BoldFont=Heiti SC,Ligatures=Rare]
\setCJKmainfont{Songti SC}[BoldFont=Heiti SC]
\usepackage[math-style=TeX]{unicode-math}
\setmathfont{Asana Math}
\fi

\ltjsetparameter{xkanjiskip=.13\zw plus 1pt minus 1pt}

\newcommand*{\enumparen}[1]{\textnormal{(}\makebox[0.5em][c]{#1}\textnormal{)}}

\usepackage{enumitem}
\setlist[enumerate,1]{label=\enumparen{\arabic*}}
\setlist[enumerate,2]{label=\enumparen{\alph*}}

\makeatletter
\def\hhline{%
  \noalign{\ifnum0=`}\fi\hrule \@height 2\arrayrulewidth \futurelet
   \reserved@a\@xhline}
\makeatother

\usepackage{float}

\begin{document}
\maketitle

此文档为《数理逻辑》(余俊伟 等,2020)一书的非官方习题解答。

\begin{description}
\item[习题 2.2.6] 前三项都是公式,剩余六项都不是公式。
\item[习题 2.2.7]
  \begin{enumerate}
  \item[]
  \item 补全括号:
    \begin{enumerate}
    \item \(((r \vee (\neg p)) \vee q)\);
    \item \(((r \to (p \to p)) \leftrightarrow ((\neg(\neg p)) \vee q))\)。
    \end{enumerate}
  \item 简化公式:
    \begin{enumerate}
    \item 题有问题,多一个左括号,或者少一个右括号。假设是后者,就有\(p \wedge q \to r\);
    \item \(p \to q\);
    \item \((r \to p \to q) \to s\)。
    \end{enumerate}
  \end{enumerate}
\item[问题 2.3.8] \(16\)和\(2^n\)。
\item[问题 2.3.12] 因为前面讨论的原子命题个数是有穷的(finite),而且现在讨论的情况是有可数无穷多个(countably infinite)原子命题。
\item[问题 2.3.21]
  \begin{enumerate}
  \item[] 
  \item 由下表可知,它是重言式。
    \begin{table}[H]
      \centering
      \begin{tabular}[t]{ccccc}
        \hhline
        \(p\) & \(q\) & \(p \to q\) & \(p \land (p \to q)\) & \(p \land (p \to q) \to q\) \\
        \hline
        1 & 1 & 1 & 1 & 1 \\
        1 & 0 & 0 & 0 & 1 \\
        0 & 1 & 1 & 0 & 1 \\
        0 & 0 & 1 & 0 & 1 \\
        \hline
      \end{tabular}
    \end{table}
  \item 由下表可知,它是矛盾式。
    \begin{table}[H]
      \centering
      \begin{tabular}[t]{cccccc}
        \hhline
        \(p\) & \(q\) & \(p \to q\) & \((p \to q) \to p\) & \(\neg p\) & \(((p \to q) \to p) \land \neg p\) \\
        \hline
        1 & 1 & 1 & 1 & 0 & 0 \\
        1 & 0 & 0 & 1 & 0 & 0 \\
        0 & 1 & 1 & 0 & 1 & 0 \\
        0 & 0 & 1 & 0 & 1 & 0 \\
        \hline
      \end{tabular}
    \end{table}
  \end{enumerate}
\item[问题 2.3.22] \(v(\neg p \land (q \to r)) = 0\)。
\item[问题 2.3.23] 对于任意的赋值\(v\),要么\(v(p) = 1\),要么\(v(p) = 0\)。在前者的情况下,有\(v(\neg p) = 0\),从而\(v(p \land \neg p) = 0\)。同理,在后者的情况下,也有\(v(p \land \neg p) = 0\)。所以它是矛盾式。\qed
\item[问题 2.3.24]
  \begin{enumerate}
  \item[]
  \item 对于任意的赋值\(v\),要么\(v(\phi) = 1\),要么\(v(\phi) = 0\)。若是前者,则\(v(\psi \to \phi) = 1\),自然有\(v(\phi \to \psi \to \phi) = 1\)。同理,若是后者,则直接有\(v(\phi \to \psi \to \phi) = 1\)。因此,它是重言式。\qed
  \item 对于任意的赋值\(v\),要么\(v(\phi \to \theta) = 1\),要么\(v(\phi \to \theta) = 0\)。若是前者,则有\(v((\phi \to \psi) \to (\phi \to \theta)) = 1\),从而\(v((\phi \to \psi \to \theta) \to (\phi \to \psi) \to (\phi \to \theta)) = 1\)。若是后者,则有\(v(\phi) = 1\)且\(v(\theta) = 0\)。此时,若\(v(\psi) = 1\),则有\(v(\phi \to \psi \to \theta) = 0\),从而\(v((\phi \to \psi \to \theta) \to (\phi \to \psi) \to (\phi \to \theta)) = 1\);若\(v(\psi) = 0\),则有\(v(\phi \to \psi) = 0\),从而有\(v((\phi \to \psi) \to (\phi \to \theta)) = 1\)和\(v((\phi \to \psi \to \theta) \to (\phi \to \psi) \to (\phi \to \theta)) = 1\)。因此,它是重言式。\qed
  \item 对于任意的赋值\(v\),要么\(v(\phi) = 1\),要么\(v(\phi) = 0\)。若是前者,则自然有\(v((\neg\phi \to \neg\psi) \to \phi) = 1\),从而\(v((\neg\phi \to \psi) \to (\neg\phi \to \neg\psi) \to \phi) = 1\)。若是后者,则\(v(\neg\phi) = 1\)。此时,若\(v(\psi) = 1\),则\(v(\neg\psi) = 0\),从而有\(v(\neg\phi \to \neg\psi) = 0\)、\(v((\neg\phi \to \neg\psi) \to \phi) = 1\)和\(v((\neg\phi \to \psi) \to (\neg\phi \to \neg\psi) \to \phi) = 1\);若\(v(\psi) = 0\),则\(v(\neg\phi \to \psi) = 0\),从而\(v((\neg\phi \to \psi) \to (\neg\phi \to \neg\psi) \to \phi) = 1\)。因此,它是重言式。\qed
  \end{enumerate}
\item[问题 2.3.25] 设\(v\)为任意满足\(v(\brce{p \to q, \neg p \to \neg q}) = 1\)的赋值函数。因为\(v(p \to q) = 1\),所以\(v(p) = 0\)或者\(v(q) = 1\)。若\(v(p) = 0\),则\(v(\neg p) = 1\)。又因为\(v(\neg p \to \neg q) = 1\),所以有\(v(\neg q) = 1\),从而有\(v(q) = 0\)和\(v(p \leftrightarrow q) = 1\)。若\(v(q) = 1\),则\(v(\neg q) = 0\)。又因为\(v(\neg p \to \neg q) = 1\),所以有\(v(\neg p) = 0\),从而有\(v(p) = 1\)和\(v(p \leftrightarrow q) = 1\)。由\(v\)的任意性可得\(\brce{p \to q, \neg p \to \neg q} \models p \leftrightarrow q\)。\qed
\end{description}

\end{document}

% Local Variables:
% TeX-engine: luatex
% End:
